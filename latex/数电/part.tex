\documentclass[11pt,twoside,a4paper]{ctexart}
\usepackage[backend=bibtex]{biblatex}
\usepackage[margin=1in]{geometry}
\usepackage{amsmath}

\begin{document}
\section{数字逻辑基础}
\subsection{模拟信号}
特点:在时间域是连续的
\subsection{数字信号}
特点:在幅度上是离散的
\subsection{模拟电路与数字电路}
两者各有自己的特点,在当今的电子设备中大多配合使用

数字电路优点:
\begin{minipage}[t]{0.9\linewidth}
    1)集成度高功耗低计算额能力强

    2)抗干扰能力强,工作可靠

    3)功能多样化,适应能力强

\end{minipage}

\subsection{数制}
需掌握:1)十进制与其他进制的相互转化 2)二进制与八,十六进制之间的特殊转化

1.十进制化其他进制:整数部分除后取余法(从下向上由高位向低位),小数部分乘后取整法(由上向下由高位向低位)

2.其他进制化十进制:展开法(不同位次的数字看作权重,乘以对应的系数,相加),其中小数部分为近似,展开的项越多近似情况越好

3.二进制化八进制:以小数点位中心,向左向右取三位为一个八进制数字化开

4.二进制化十六进制:以小数点位中心,向左向右取四位为一个八进制数字化开

\subsection{编码(代码)}
\subsubsection{二-十进制代码}
被编码的信息量为M,则对用于编码的二进制数的位数n有:$n \geq log_2(M) $

1.8421BCD码:用四位二进制数字表示一个十进制的数字,是一种有权码:按照十进递增规律编码,$ 1010 \sim  1111 $为禁用码

\[(0101 1000)_8421BCD = (58)_10  \]

显然用8421BCD码化二进制时要先化十进制

\subsubsection{格雷码}
1.任意两个相邻码之间只有一位不同,在传输过程中引起的误差小。2.是一种无权码。3.二进制和格雷码无法建立联系

\subsubsection{字符代码}

\subsection{带符号二进数的表示方法}
\subsubsection{原码}
一个十进制数对应的二进制数为他的原码:$(13)_10 = (1101)_2 $,1101: 原码
\subsubsection{反码}
将原码中的1变0,0变1:1101的反码为0010
\subsubsection{补码}
给反码加一:$0010 + 1 = 0011 $,则0011是0010的补码

原码直接化补码:从右侧数第一个1不变,其余数字取反,$1101 \rightarrow 0011 $
\subsubsection{正负数表示}
最左位增加一个符号位

\[ \text{符号位} 
\begin{cases}
0 \text{表示正数} \\
1 \text{表示负数}
\end{cases} \]

正数:原码 = 反码 = 补码,例:$(13)_10 $原码反码补码都是01101

负数:原码反码补码前加一,例:$-(13)_10 $原码:11101,反码:10010,补码:10011

用补码做减法可以把减法变加法$A - B \rightarrow A + (-B) $,其中A用原码,B用补码,则计算机运算A的原码加B的补码

\subsubsection{偏移码}
补码的符号位取反


\end{document}
