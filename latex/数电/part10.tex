\documentclass[11pt,twoside,a4paper]{ctexart}
\usepackage[backend=bibtex]{biblatex}
\usepackage[margin=1in]{geometry}
\usepackage{amsmath}

\begin{document}
    
\section{半导体存储器及可编程逻辑器件}

\subsection{半导体存储器分类}

1.按照存储方式:1)SAM顺序存储器,2)RAM随机存储器,3)ROM只读存储器

2.按照基本电路:1),2)MOS

\subsection{技术参数}

1.存储周期:两次读(写)操作之间的间隔,越短越好

2.存储容量:字节Byte,位Bit,容量位字数乘位数$2k*16$

\subsection{随机存取器RAM}

\subsubsection{RAM的结构}

1.存储矩阵

2.地址译码器

3.输入输出控制

\subsubsection{RAM芯片}

\subsubsection{RAm拓展}

1.位拓展(加每个宿舍床个数):公用地址线,读写线,片选线,增加数据线

2.字拓展(加宿舍房间个数):增加地址线,用片选线扩展,其余公用

3.字位同时扩展:先扩位再扩字


\end{document}