\documentclass[11pt,twoside,a4paper]{ctexart}
\usepackage[backend=bibtex]{biblatex}
\usepackage[margin=1in]{geometry}
\usepackage{amsmath}

\begin{document}
    \section{逻辑门电路}
    \subsection{逻辑门电路}
    \subsubsection{基本逻辑门电路}
    1.与门

    1)结构:两个或以上输入,一个输出

    2)逻辑函数:$F = A\cdot B $

    3)功能:当输入全为高电平时输出高电平,其余全输出低电平

    4)运算类比集合:将1视为全集,0视为空集,则有:将$\cdot $ 看作集合运算中的$\cap $ 

    2.或门

    1)结构:两个或以上输入,一个输出

    2)逻辑函数:$F = A + B $

    3)功能:当输入有高电平时输出高电平,输入没有高电平时输出低电平

    4)运算类比集合:将1视为全集,0视为空集,则有:$ + $看作集合运算中的$\cup $ 
    
    3.非门

    1)结构:一个输入,一个输出

    2)逻辑函数:$F =\overline{A}$

    3)功能:当输入有高电平时输出低电平,输入低电平时输出高电平

    4)运算类比集合:将1视为全集,0视为空集,则有:看作集合运算中的取补集

    note:记忆所有的逻辑符号课本P15

    \subsubsection{复合逻辑门电路}
    1.与非门:先与运算,再非运算,$F = \overline{A\cdot B} $

    2.或非门:先或运算,再非运算,$F = \overline{A + B} $

    3.与或非门:先与运算或运算,再非运算,$F = \overline{A\cdot B + C\cdot D} $

    4.异或门:当输出相同时输出低电平,输入不同时输出高电平,$F = A\oplus B = A\overline{B} + \overline{A}B $

    5.同或门:异或门的结果再进行非运算得到,$F = A\otimes B = \overline{A\overline{B} + \overline{A}B} = \overline{A\cdot B} + A\cdot B $


    \subsection{TTL集成门电路}
    此部分为理论推导,课本p17看详细过程

    记忆:TTL带负载门最多8个

    \subsubsection{门槛电压}
    仅对ttl由这个作用,在输入端连接接地的门槛电阻$R_i$,临接电压$R_T = 2k\Omega $

    当$R_i < R_T$时,等效输入低电平

    当$R_i > R_T$时,等效输入高电平

    对地悬空($R_i = \infty$)视为逻辑高电平

    \subsubsection{三态门}

    1.高电平有效:
    
    EN:使能输入端;

    EN = 1,F = $\overline{AB} $

    EN = 0,F = Hi - Z (悬空)
    
    2.低电平有效

    EN = 1,F = Hi - Z (悬空)

    EN = 0,F = $\overline{AB} $
    \subsection{MOS逻辑电路}
    课本p28





\end{document}
