\documentclass[11pt,twoside,a4paper]{ctexart}
\usepackage[backend=bibtex]{biblatex}
\usepackage[margin=1in]{geometry}
\usepackage{amsmath}

\begin{document}
    \section{逻辑代数基础}
    \subsection{逻辑代数运算法则}
    \subsubsection{基本逻辑运算}
    1.与
    
    2.或

    3.非
    \subsubsection{逻辑代数基本定理}
    同离散数学,新增0 1,将0理解为空集,1理解为全集即可

    \subsubsection{逻辑代数的基本规则}
    1.代入

    2.反演:对于一个逻辑函数F,将其中的变量取反,与运算变或运算,或运算变
    与运算,0换1,1换0,最后得到的函数$\overline{F}$依然成立,为F的反函数

    3.对偶:对于一个逻辑函数F,与运算变或运算,或运算变
    与运算,0换1,1换0,最后得到的函数$F'$依然成立,为F的对偶式
    $(F')' = F$
    \subsubsection{常用公式}
    p43,习题熟悉
    \subsection{逻辑函数的标准形式}
    \subsubsection{最小项和标准与或式}
    和离散数学中的主析取范式完全相同
    1.最小项:多个变量取乘积为与项,每个变量和其否定不同时出现,n个变量有$2^n$个最小项

    2.标准与或式(主析取范式):每项都是最小项的与或式
    \subsubsection{最大项和标准或与式}
    同离散数学主合取范式
    \subsubsection{最大项和最小项的关系}
    最大项和最小项互补:$\overline{m_i} = M_i$

    主析取范式变为主合取范式过程:

    \ 1.求出主析取范式中没有小项

    \ 2.求出和1中小项相同下标的大项

    \ 3.2中所有大项合取,就是所求对应的主合取范式
    \subsection{逻辑函数的公式化简法}
    最简逻辑函数在连接电路时最简单,节省材料,所以需要将原函数简化。

    最简表达式:与-或表达式,或-与表达式,与非-与非,或非-或非,与-或-非,或-与-非

    要求:项数最少,每项变量个数最少
    \subsection{卡诺图化简法}
    \subsection{卡诺图}
    

    


\end{document}