\documentclass[11pt,twoside,a4paper]{ctexart}
\usepackage[backend=bibtex]{biblatex}
\usepackage[margin=1in]{geometry}
\usepackage{amsmath}

\begin{document}
    \section{组合逻辑电路}
    1.特点:1)输出是输入的逻辑运算组合,输出在任何时刻只和输入有关
    2)不具有反馈电路或记忆,延迟单元,电路中的信息单向传递
    \subsection{组合逻辑电路分析}
    1.分析步骤:
    \begin{minipage}[t]{0.9\linewidth}
        1)从每一级写出输入输出

        2)写出各个器件的输出函数表达式并合并

        3)列逻辑函数的真值表达式

        4)分析逻辑功能(功能可能不是单一的

    \end{minipage}
    \subsection{组合逻辑电路设计}
    为电路分析的逆过程,有一个目标反推出相应电路
    \subsection{编码器}
    用于把输入的信号编成二进制代码,一般N个信号,至少需要n位二进制数来编码
    $2^n \geq N$
    \subsubsection{普通编码器}
    1.8线-3线编码器

    输入8个信号(0~7),输出3(2 1 0)个信号,且输入信号互斥,在相同时刻只能由一个信号输入。

    输入高电平信号转化为三位的二进制为多少,则输出对应表现为多少的高电平二进制

    \subsection{译码器}
    与编码过程相反,输入二进制代码,以标准形式输出特定的高低电平

    使能端:决定整个译码器系统是否工作的输入,只有满足特定输入时译码器才能工作
    使用不同的门电路来构建译码器电路时,使能端也会有所不同
    \subsubsection{二进制编码器}
    1.2线-4线译码器

    2个输入端,作为二进制可以表示0,1,2,3一共四位,所以可以有四位标准形式的输出

    1)输出高电平有效

    输出高电平有效0001,0010,0100,1000

    2)输出低电平有效

    输出低电平有效1110,1101,1011,0111

    2.3线-8线译码器

    输出低电平有效 IC74138

    三位二进制输入,可以表示十进制0,1,2,3,4,5,6,7一共八位

    分别对应八个输出端的其中一个

    使能端:$S_1,\overline{S_2}\overline{S_3},$固定输入100时译码器工作,其余锁死

    3.译码器实现逻辑函数

    思路:译码器可以输出标准形式,将逻辑函数化为标准形式,在配合或门或者与门,
    就可以使用译码器实现逻辑函数

    步骤:1)将逻辑函数化为标准形式;2)将译码器的输出连接到对应的与,或门上;3)对标准形式进行变换,可以将与,或门换为其他门

    总结:1)高电平有效译码器+或门(最小项)2)低电平有效译码器+与门最大项(与非门最小项)

    \subsubsection{码制变换译码器}

    将一种进制变换为另一种进制,例如IC7447,4线-10线译码器

    1.4线-10线译码器

    输入8421BCD码,输出低电平有效十进制数字0~9.

    \subsubsection{显示译码器}

    1.七段数码管

    利用七段灯的亮灭表示数字,根据连接方式分为两种

    1)共阴极:阳极为高电平亮

    2)共阳极:阴极为低电平亮

    2.显示译码器 7448
    
    七段数码管在使用时,和译码器配合使用,译码器作为驱动。

    7448:4线-7线,输出高有效,驱动共阴极管。

    可以通过卡诺图来写出每一个数码管的逻辑函数

    \subsection{数据选择器}

    MUX

    在输入的多线数据中选择一个作为输出

    \subsubsection{数据选择器举例}

    1.4线-1线MUX

    输入信号有四个$D_0,\cdots,D_3$;两个地址信号$A,B$;使能端$\overline{E}$

    A,B以二进制表示选择哪一个信号

    2.8线-1线MUX

    3位地址线,8位数据线

    3.MUX实现逻辑函数

    思路:查询对应最小项是否为1。

    实现方法:例如三变量逻辑函数,选择8线-1线MUX,将变量对应连接到地址线,在逻辑函数
    对应最小项的数据线输入1,其余输入0;则在变量输入到对应最小项时,输出为1,其余时刻输出为0,符合逻辑函数

    \subsection{数据分配器}

    DEMUX

    为数据选择器的逆过程

    有一个数据输入线,n个地址线,$2^n$个数据输出线,由地址线决定输入由哪一条输出

    将MUX和DEMUX配合使用,可以得到多路数字开关。

    \subsection{数值比较器}

    \subsubsection{1位数值比较器}

    2线输入,3线输出

    1.输入两个一位数字A,B;输出L(A>B),E(A=B),S(A<B)

    2.逻辑函数$L = A\overline{B},E = A\odot B,S = \overline{A}B $
    
    \subsubsection{4位数值比较器}

    7485

    7线输入,其中4线数字输入,3线级联输入用来表述低位的大小结果,三线输出

    从高位开始比较,若高位能得到大小不同,直接输出,若高位大小相同再比较低位

    \subsubsection{比较器级联扩展}

    若要比较8位数字,使用两个4位7485比较器串联,将低位片的输出端作为级联输入连接到高位片

    输出在高位片输入能区别大小时,低位片输出不作用,高位相等时,低位片输出决定最终结果

    \subsection{加法器}

    \subsubsection{半加器}

    进行两位二进制数字的加分,不考虑低位进位。

    输入两个二进制数字A,B;输出本位S,进位C

    逻辑函数$S = A\oplus B,C = AB$

    \subsubsection{全加器}

    考虑来自低位的进位$C_i$,输入两个二进制数字A,B,输出本位S,进位$C_{i+1}$

    逻辑函数:$S = A\oplus B\oplus C_i,C_{i=1} = AB + AC_i + BC_i$

    在进行多位二进制数字的加法时,每位使用一个全加器,将进位输出连接到更高位的进位输出,构成并行输出,串行进位的加法器
    但是这种加法器的速度较慢

    \subsubsection{超前进位加法器}
    课本P82,可以实现并行进位的计算

    \subsection{组合逻辑电路的竞争冒险}

    1.竞争:从输入到输出的途径不同,延时时间不同,到达输出端的时间不同,这种现象为竞争

    2.冒险: 竞争结果导致逻辑电路产生错误输出,称为冒险或险象

    \subsubsection{竞争冒险消除方法}

    1,接入滤波电容

    2.引入取样脉冲

    3.修改设计方案



    

\end{document}