\documentclass[11pt,twoside,a4paper]{ctexart}
\usepackage[backend=bibtex]{biblatex}
\usepackage[margin=1in]{geometry}
\usepackage{amsmath}

\begin{document}
    \section{组合逻辑电路}
    1.特点:1)输出是输入的逻辑运算组合,输出在任何时刻只和输入有关
    2)不具有反馈电路或记忆,延迟单元,电路中的信息单向传递
    \subsection{组合逻辑电路分析}
    1.分析步骤:
    \begin{minipage}[t]{0.9\linewidth}
        1)从每一级写出输入输出

        2)写出各个器件的输出函数表达式并合并

        3)列逻辑函数的真值表达式

        4)分析逻辑功能(功能可能不是单一的

    \end{minipage}
    \subsection{组合逻辑电路设计}
    为电路分析的逆过程,有一个目标反推出相应电路
    \subsection{编码器}
    用于把输入的信号编成二进制代码,一般N个信号,至少需要n位二进制数来编码
    $2^n \geq N$

\end{document}