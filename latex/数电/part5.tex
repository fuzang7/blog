\documentclass[11pt,twoside,a4paper]{ctexart}
\usepackage[backend=bibtex]{biblatex}
\usepackage[margin=1in]{geometry}
\usepackage{amsmath}

\begin{document}
    \section{触发器}
    1.触发器:具有记忆功能的基本逻辑单元,可以储存一位二值信号

    2.特点: 
     1)能保持稳定的状态

     2)在外加信号的影响下能够改变状态,而且可以保持新状态
    
    \subsection{电平触发的触发器}
    \subsubsection{与非门构成的基本RS触发器}
    1.锁存器 p93

    2.有两个输入端$\overline{S},\overline{R}$,前者为置1端后者为置0端
    且为低电平有效,要求$\overline{S},\overline{R}$不能同时为0.两个输出端
    $Q,\overline{Q}$。Q的状态为触发器的状态。

    3.$\overline{S},\overline{Q}$同时为0时,触发器的状态由门的延迟决定,是不确定状态

    \subsubsection{触发器逻辑功能的描述方法}
    1)状态转移真值表:以输入端和现在的稳定状态为输入,下一状态的稳定状态为输出

    2)状态方程:由卡诺图化简而成的逻辑函数,切记要写要求

    3)状态转移图:由图表示的状态转移,在箭头上写转移的要求

    4)激励表:触发器的状态由上一时刻到下一时刻要求激励输入时什么

    5)波形图
    \subsubsection{或非门组成的基本RS触发器}
    p96

    \subsubsection{基本RS触发器特点}
    状态由激励输入直接决定,没有统一的控制信号控制触发器的转换时刻,为异步时序电路

    \subsection{时钟触发器}

\end{document}