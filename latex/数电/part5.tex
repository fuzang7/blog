\documentclass[11pt,twoside,a4paper]{ctexart}
\usepackage[backend=bibtex]{biblatex}
\usepackage[margin=1in]{geometry}
\usepackage{amsmath}

\begin{document}
    \section{触发器}
    1.触发器:具有记忆功能的基本逻辑单元,可以储存一位二值信号

    2.特点: 
     1)能保持稳定的状态

     2)在外加信号的影响下能够改变状态,而且可以保持新状态
    
    \subsection{电平触发的触发器}
    \subsubsection{与非门构成的基本RS触发器}
    1.有两个输入端$\overline{S},\overline{R}$,前者为置1端后者为置0端
    且为低电平有效,要求$\overline{S},\overline{R}$不能同时为0.两个输出端
    $Q,\overline{Q}$。Q的状态为触发器的状态。

    2.$\overline{S},\overline{Q}$同时为0时,触发器的状态由门的延迟决定,是不确定状态

    \subsubsection{触发器逻辑功能的描述方法}
    1)状态转移真值表:以输入端和现在的稳定状态为输入,下一状态的稳定状态为输出

    2)状态方程:由卡诺图化简而成的逻辑函数,切记要写要求

    3)状态转移图:由图表示的状态转移,在箭头上写转移的要求

    4)激励表:触发器的状态由上一时刻到下一时刻要求激励输入时什么

    5)波形图
    \subsubsection{或非门组成的基本RS触发器}
    高电平有效

    \subsubsection{基本RS触发器特点}
    1.输入信号在所有时间都能改变输出的状态
    
    2.没有统一的控制信号控制触发器的转换时刻,为异步时序电路

    3.有输入条件的限制,存在非法输入

    \subsubsection{时钟触发器}
    为了统一动作引入时钟CLK作为控制,只允许在$CLK = 1$的时候Q发生改变

    1.时钟RS触发器

    高电平有效,加入了CLk信号作为控制

    2.时钟D触发器

    使用一个信号D同时控制RS,还能满足RS = 0的要求

    3.时钟JK触发器

    克服了$RS = 0$的限制,增加了反转功能

    4.时钟T触发器

    仅有反转功能

    5.时钟触发器的特点

    1)在整个时钟$CLK = 1$的期间,输入信号都可以影响到输出信号

    2)可能出现 空翻:在一个CLK期间触发器状态Q变化了不止一次

    \subsection{脉冲触发的触发器}

    \subsubsection{主从RS触发器}
    要求在每个CLK脉冲期间状态只能变化一次,主从RS触发器在正常工作时,需要一个完整的脉冲周期:时钟从0到1再回到0才能完成状态的改变

    1.电路:使用了两个时钟RS触发器,且两个触发器之间的时钟置反,则只有CLK完成 0-1-0 的一个完整脉冲周期,才能使结果发生改变

    2.下边沿触发器完成次态的转移

    \subsubsection{主从JK触发器}

    1.解决了输入限制,增加了反转功能

    2.下边沿触发器完成次态的转移

    3.有直接接入的异步信号$\overline{S}_D,\overline{R}_D$输入,可以无视时钟直接改变触发器的输出状态

    \subsubsection{主从D触发器}

    1.使用一个信号控制主从JK触发器的两个输入,是主从JK触发器中的$J \neq K$的部分

    \subsubsection{主从T触发器}

    1.是主从JK触发器中的$J = K$的部分,仅有反转功能

    \subsubsection{一次变化}

    主从JK触发器在$CLK = 1$期间,输入信号若是发生变化,则会发生一次变化现象,表现为逻辑关系的破坏,因此只能在CLK窄脉冲触发时使用主从JK触发器

    \subsubsection{时序波形图的注意点}

    1.异步置1端和异步置0端优先级高于时钟

    2.如果异步信号无效,在时钟信号的有效沿按输入信号确定触发器状态。如果时钟有效沿和输入信号的变化同时发生,取时钟有效沿前的信号确定触发器的状态

    3.如果异步信号从有效变无效的时刻和时钟有效沿重合,则当前时钟有效沿失效,按照之前瞬间异步信号确定触发器状态

    \subsection{边沿触发的触发器}

    希望触发器的次态仅取决于CLK的下降沿(或上升沿)到达前输入信号的状态,边沿触发的触发器不要求一个完整的脉冲周期就能实现状态的转换

    \subsubsection{维持阻塞D触发器}

    上升沿有效,输出D

    \subsection{触发器之间的转换}
    p112

\end{document}