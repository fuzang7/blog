\documentclass[11pt,twoside,a4paper]{ctexart}
\usepackage[backend=bibtex]{biblatex}
\usepackage[margin=1in]{geometry}
\usepackage{amsmath}
\begin{document}
    
\section{时序逻辑电路}
\subsection{时序逻辑电路的基本概念}
\subsubsection{时序逻辑电路的结构及特点}

1.特点:时序逻辑电路在任何时刻的输出状态同时取决于输入信号与电路的原状态

2.结构:时序逻辑一般由组合逻辑电路以及触发器电路组成

3.外输入X,外输出Z,控制输入W,触发器输出(状态)Q

\subsubsection{时序逻辑电路的表示方法}

1.方差组表示

1)输出方程:$Z = F(X,Q)$

2)驱动方程:$W = H(X,Q)$

3)特征方程:$Q^{n+1} = G(W,Q^n)$

2.状态图和状态表

1)状态表:电路中所有状态作为现态列在表的左边,对应的次态和输出在右边

2)状态图:用来描述不同状态之间的转换

3.时序图

\subsubsection{时序逻辑电路的分类}

根据输出变量是否和输入变量直接相关分为米里型(Z与Q,X相关),莫尔型(Z与Q相关)

\subsection{时序逻辑电路分析}
根据电路分析功能

1.步骤:
1)确定变量

2)写出方程

3)状态表和状态图

4)电路功能

\subsection{时序逻辑电路的设计}


\end{document}