\documentclass[11pt,twoside,a4paper]{ctexart}
\usepackage[backend=bibtex]{biblatex}
\usepackage[margin=1in]{geometry}
\usepackage{amsmath}

\begin{document}
    
\section{数模转换和模数转换}
\subsection{数模转换电路DAC}
\subsubsection{数模转换关系}

1.单极性码:例如自然加权二进制码,不表示正负

2.双极性码:可以表示正负极性的模拟信号,例如原码,补码,偏移码(补码的符号位取反得到),使用双极性码时其满刻度是单极性码满刻度值的一半

3.归一化表示法:以一一对应的方式来转换输入的二进制代码,每一个二进制数都转换成满刻度值(FSR)的一个确定的分数

4.最低有效位LSB对应$\frac{1}{2^n}FSR$,其中n是数字量的位数

\subsubsection{权电阻网络DAC}

1.参考电压$V_{ref}$;从高位到低位的数字量$X_1,X_2,X_3$,分别控制开关$S_1,S_2,S_3$

2.电路结构p197,由叠加定律,模拟输出电压$V_0$
\[V_o = -iR_f = -\frac{2R_f}{R}V_{ref}*\frac{X_12^2 + X_22^1 + X_32^0}{2^3}\]
其中$-\frac{2R_f}{R}V_{ref}$为FSR,分子上是输入的二进制数字展开的十进制数,切勿忘记反向

3.分辨率$s = |V_{Omin}| = \frac{1}{2^3}FSR$

\subsubsection{R-2R梯形电阻网络DAC}

1.电路结构p198

2.各个连接点对地电阻都是R,且整个网络中只有两种阻值的电阻

3.根据戴维南定律推导
\[V_o = -\frac{R_f}{R}V_{ref} * \frac{X_12^2 + X_22^1 + X_32^0}{2^3}\]
其中$\frac{R_f}{R}V_{ref}$是FSR,切记与权电阻网络的不同;最大值$-\frac{7}{2^3}FSR$;最小值$-\frac{1}{2^3}FSR$

\subsubsection{R-2R倒梯形电阻网络DAC}

1.电路结构p199

2.
\[V_o = -\frac{R_f}{R}V_{ref} * \frac{X_12^2 + X_22^1 + X_32^0}{2^3}\]

\subsubsection{集成数模转换电路}

0.讨论双极性码,补码相对原码更好,因为其0只有一种表达;偏移码相对补码更好,因为具有模拟量从小到大,偏移码也从小到大的独有特点

偏移码其实就相当于将坐标系偏移到单极性码的中心时的表现

1.10位CMOS集成DAC -- AD7533

使用时要外接参考电压$V_{ref}$和运算放大器,把电流输出转换为电压输出

1)结构见p201

2)
\[I_{OUT1} = I_{ref}(X_12^{-1} + X_22^{-2} + \cdots + X_{10}2^{-10})\]
\[I_{OUT2} = I_{ref}(\overline{X}_12^{-1} + \overline{X}_22^{-2} + \cdots +\overline{X}_{10}2^{-10})\]
\[I_{OUT1} + I_{OUT2} = \frac{1023}{1024}I_{ref}\]

3)接收自然加权码

\begin{align}
    V_o & =-V_o = I_{OUT1}R = I_{ref}R\frac{(X_12^{-1} + X_22^{-2} + \cdots + X_{10}2^{-10})}{2^{10}} \\
    & = V_{ref}\frac{(X_12^{-1} + X_22^{-2} + \cdots + X_{10}2^{-10})}{2^{10}}
\end{align}
其中$V_{ref} = I_{ref}R$是FSR

4)接受偏移码
\[V_o = -V'_o = (I_{OUT1} - \frac{I_{ref}}{2})R = (X_1\frac{I_{ref}}{2} + X_2\frac{I_{ref}}{2^2} + \cdots + X_{10}\frac{I_{ref}}{2^{10}} - \frac{I_{ref}}{2})R\]

5)接受补码

将补码的符号位取反,然后和偏移码相同
\[V_o = -V'_o = (I_{OUT1} - \frac{I_{ref}}{2})R = (\overline{X}_1\frac{I_{ref}}{2} + X_2\frac{I_{ref}}{2^2} + \cdots + X_{10}\frac{I_{ref}}{2^{10}} - \frac{I_{ref}}{2})R\]

\subsection{ADC模数转换电路}

\subsubsection{量化}

1.量化:将模拟量离散化的过程

2.量化单位s:在量化过程中的最小数量单位,也叫做量化阶梯

3.量化误差$\varepsilon $:取样电压不一定能被s整除,量化前后必然出现误差,此误差无法消除

4.量化方法:只舍不入量化方式,四舍五入量化方式

5.四舍五入法:$s = \frac{1}{2^n - 1}$,误差较小

6.只舍不入量化法:$s = \frac{1}{2^n}$

\subsubsection{并行比较ADC}

\subsubsection{并/串型ADC}

可以克服并行的硬件繁多

将输入的高低位分开处理,高位进行只舍不入的AD转换,低位进行四舍五入的AD转换,最后统合起来得到结果

\subsubsection{逐次逼近型ADC}

1.类似于天平称量,从最高位开始比较,通过多次比较输入信号与二进制信号的大小,最后得到相对最近的二进制信号作为输出

2.输入只舍不如ADC

3.完成一次转换需要的时间$t = (n + 2)T_{CLK}$

\subsubsection{双积分ADC}

1.进行两次积分,第一次,使用输入电压进行定时积分,通过固定$t_1$时长的积分,得到一个电压值保存,在此过程中计数器收到$(2^n - 1)$个CLK;在第$2^n$个CLK,计数器复0

2.第二次,使用$-V_{ref}$进行定压积分,计数器计时直到电压在积分的作用下复0,记此时计数器接到N个时钟脉冲
\[N = \frac{\overline{V}_{in}}{V_{ref}}2^n\]
则十进制N与输入电压成正比,实现了A/D转换

\end{document}