\documentclass[11pt,twoside,a4paper]{ctexart}
\usepackage[backend=bibtex]{biblatex}
\usepackage[margin=1in]{geometry}
\usepackage{amsmath}
\usepackage{amssymb}

\begin{document}
    \section{概率论的基本概念}
    \subsection{随机事件及其运算}
    \subsubsection{随机实验和随机事件}
    1.随机实验条件:


    \begin{quote}
        1)可重复性


        2)结果有多个且已知 

        
        3)结果不可预知 
    \end{quote}


    2.
    \begin{minipage}[t]{0.9\linewidth}
        样本空间:$\Omega = \{\omega \} $ 所有可能结果的集合

        样本点:$\Omega $ 中的一个元素

        随机事件:
        \begin{minipage}[t]{0.8\linewidth}
            集合$\Omega $ 的子集,其中$\Omega $ 本身为必然事件,$\emptyset $ 为不可能事件

            当事件 A 中的样本点出现称为事件 A 发生
        \end{minipage}
    \end{minipage}
    

    \subsubsection{事件间的关系和运算}
    1.事件的包含与相等:同集合的包含与相等

    2.和事件(并集): $A \cup B \iff$ A和B至少有一件发生

    3.积事件(交集): $A \cap  B = AB \iff$ A和B同时发生

    4.差事件(差集): $A - B \iff$ A发生但B不发生

    5.补事件(补集): $\overline{A} = \Omega - A $ , $A - B = A - AB = A\overline{B}$

    6.互不相容(互斥)
    \subsection{概率的定义及其基本性质}
    \subsubsection{频率和概率}
    1.频率:$f_n(A) = \frac{n_A}{n}  $ ,实验重复n次,其中事件A发生的次数为$n_A $

    \quad 性质:
    \begin{minipage}[t]{0.9\linewidth}
        1)非负性:$f_n(A) \geq 0 $

        2)归一性:$f_n(\Omega) = 1 $

        3)可列可加性:若$A_1 , A_2 , \cdots , A_k $为两两不相容事件,则:\[f_n(\bigcup _{i=1}^k A_i) = \sum_{i=1}^k f_n(A_i)  \]
        
    \end{minipage}

    2.概率:当随机实验重复的次数足够多时,事件A发生的频率将趋近于一个常数,这个常数就是事件A发生的概率,记作P(A)

    概率公理化定义:
    \begin{minipage}[t]{0.9\linewidth}
        随机实验中,对于事件A,有一个常数P(A)满足:

        1)非负性:$P(A) \geq 0 $

        2)归一性:$P(\Omega) = 1 $

        3)可列可加性:若$A_1 , A_2 , \cdots , A_k $为两两不相容事件,则有:\[P(\bigcup _{i=1}^k A_i) = \sum_{i=1}^k P(A_i)  \]
    
    \end{minipage}
    则称P(A)为事件A的概率
    
    概率的性质:
    \begin{minipage}[t]{0.9\linewidth}
        1)$P(\varPhi ) = 0 $

        2)有限可加:若$A_1 , A_2 , \cdots , A_n $为两两不相容事件,则有:\[P(\bigcup _{i=1}^n A_i) = \sum_{i=1}^n P(A_i)  \]

        3)若$A\subset B $,则$P(A) \le P(B) $

        4)$P(A) + P(\overline{A}) = 1  $

        5)$P(A - B) = P(A\overline{B}) = P(A) - P(AB) $

        6)$P(A + B) = P(A) + P(B) - P(AB) $
        \ 更多项相加时记忆口诀:单加双减
    \end{minipage}
    \subsection{等可能概型}
    \subsubsection{古典概型}
    1.定义:
    \begin{minipage}[t]{0.9\linewidth}
        1)样本空间中的样本的总数有限

        2)每个样本点出现的可能性相等

    \end{minipage}

    2.计数方法(课本p12页例题):
    \begin{minipage}[t]{0.9\linewidth}
        1)取球问题

        2)分房(分组)问题

        3)抽签问题

        4)生日问题

    \end{minipage}

    \subsubsection{几何概型}
    1.定义:
    \begin{minipage}[t]{0.9\linewidth}
        1)样本空间中的样本的总数无限

        2)每个样本点出现的可能性相等

    \end{minipage}

    常利用长度面积体积的比来求概率

    2.计数方法:
    \begin{minipage}[t]{0.9\linewidth}
        1)会面问题

    \end{minipage}
    \subsection{条件概率}
    \subsubsection{条件概率的定义}
    1.定义:在已知事件B发生的情况下事件A发生的概率记为$P(A\vert B) $

    2.性质:
    \begin{minipage}[t]{0.9\linewidth}
        1)非负性:$P(A\vert B) \geq 0 $

        2)归一性:$P(\Omega \vert B) = 1 $

        3)可列可加性:若$A_1 , A_2 , \cdots  $为两两不相容事件,则有:\[P(\bigcup _{i=1}^\infty A_i\vert B) = \sum_{i=1}^\infty P(A_i\vert B)  \]

    \end{minipage}
    \subsubsection{乘法公式}
    乘法公式:\[P(B\vert A) = \frac{P(AB)}{P(A)} \]

    \subsubsection{全概率公式和贝叶斯公式}
    1.全概率公式:设$A_1 , A_2 , \cdot , A_n $是$\varOmega $的一个划分,则是对任意的事件B有:
    \[P(B) = \sum_{i=1}^n {P(A_i)P(B\vert A_i)  }  \] 

    在此处n可以取$\infty $

    使用全概率公式的情况:1)一个随机试验包括两个阶段,其中第二个阶段的结果收第一个阶段的影响. 2)第一个阶段的所有可能结果已知.3)求第二个阶段的概率。将第一个阶段的各种情况看作是划分

    2.贝叶斯公式:设$A_1 , A_2 , \cdot , A_n $是$\varOmega $的一个划分,如果$P(A_k) > 0 $,则对认识的事件B,只要$P(B) > 0 $,就有

    \[P(A_k\vert B) = \frac{P(A_kB)}{P(B)} = \frac{P(A_k)P(B\vert A_k)}{\sum_{i=1}^n {P(A_i)P(B\vert A_i)  }} \]

    \subsection{独立性和伯努利试验}
    \subsubsection{独立性}
    1.定义:对于事件A和事件B,当$P(AB) = P(A)P(B) $时,A和B相互独立。该式子是证明独立性的唯一条件。

    note:
    \begin{minipage}[t]{0.9\linewidth}
        1)$P(AB) = P(A)P(B)  \nLeftrightarrow  P(B\vert A) = P(B) $,后者要求$P(A) > 0 $,用前者作为定义显然适用性更广

        2)独立性表明A对B的概率没有影响,而不是说A对B没有影响

        3)区别两两独立和相互独立,两两独立仅是所有两者组合独立,而相互独立包括三者及以上的独立性,所以相对独立条件包括两两独立,为更强条件

    \end{minipage}

    2.性质:
    \begin{minipage}[t]{0.9\linewidth}
        1)独立事件必然相容,且在每一部分的概率成比例

        2)事件A与自身独立,则$P(A) = 1 \quad or \quad P(A) = 0  $

        3)在$(A,B), (\overline{A},B),(A,\overline{B}),(\overline{A},\overline{B}) $之中,有一对相互独立,则其他几组也都相互独立

    \end{minipage}
    \subsubsection{n重伯努利试验}
    1.定义:一个试验E仅有两种结果$A,\overline{A} $,并且$P(A) = p,P(\overline{A}) = 1 - p = q (0 < p < 1) $而对这个试验重复n次,这n次试验相互独立,则将这n次试验看作一次试验,称为n重伯努利试验。
    
    求事件A出现m次的概率:
    \[P(A_m) = C_n^m P(A)^m P(\overline{A})^{n-m} = C_n^m p^m q^{n-m} \] 



    



    


    
\end{document}