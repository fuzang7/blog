\documentclass[11pt,twoside,a4paper]{ctexart}
\usepackage[backend=bibtex]{biblatex}
\usepackage[margin=1in]{geometry}
\usepackage{amsmath}
\usepackage{amssymb}
\begin{document}
    \section{随机变量及其分布}
    \subsection{随机变量及其分布函数}
    \subsubsection{随机变量}
    1.定义:设$\varOmega  $为一个样本空间,若对任意的$\omega \subset \varOmega $,
    都有一个实数$X(\omega) $于之对应,那么称$X(\omega) $是一个随机变量
    \subsubsection{随机变量的分布函数}
    1.定义:
    \[F(x) = P(X \leq x),x \in R \]
    $F(x)$是随机变量的分布函数

    2.随机变量与高等函数中所学函数的不同:随机变量无法定义极限,
    也就无法定义连续性和求导

    3.分布于函数的特点:每个随机变量都由唯一的分布函数,并且一维随机变量的概率特性
    完全由它的分布函数来决定

    4.分布函数的性质
    \[\begin{cases}
        \text{(1)右连续  }  \\
        \text{(2)单调不减  }  \\
        \text{(3)}F(-\infty) = 0,F(+\infty) = 1  \\  

    \end{cases}\]
    当一个函数$g(x) $满足上述三个性质,则必然存在一个随机变量,使g为其分布函数

    5.推导公式
    \begin{gather}
        P(a\leq X \leq b) = P(b) - P(a - 0)  \notag \\
        P(a < X < b) = P(b - 0) - P(a)   \notag \\
        P(a\leq X < b) = P(b - 0) - P(a - 0)   \notag 
    \end{gather}

    \subsection{离散型随机变量}
    \subsubsection{分布列及其性质}
    1.定义:随机变量X的取值为一些离散的值(可列多个),
    那么X就是一个离散型随机变量
    \[P(X = x_i) = p_i \]
    为X的分布列

    2.特点:离散型随机变量的分布函数和分布列等价,而且离散型随机变量
    的分布函数一定是阶梯函数,反过来也成立,一个随机变量的分布函数是
    阶梯函数,那么它一定是离散型随机变量

    3.性质:
    \begin{minipage}[t]{0.9\linewidth}
        1)非负性:$p_i \geq 0 $

        2)归一性:$\sum_{i=1}^\infty p_i = 1 $

    \end{minipage}

    \subsubsection{常见的离散型随机变量}
    1.二项分布:将样本空间$\varOmega  $以$A,\overline{A} $做划分,$P(A) = p,1 - P(A) = q $
    设离散型随机变量X为在n次独立重复的试验中事件A发生的次数,则:
    \[P(X = k) = C_n^kp^kq^{n-k},\quad k = 0,1,\cdots,n \]
    称X服从参数为n的二项分布,记为$X~B(n,p) $

    note:1)0-1分布:$X~B(1,p) $,只进行一次随机试验

    2.泊松分布:对于稀有事件,其发生的概率p很小,但是试验的次数n很大时,
    设离散型随机变量X为在n次独立重复的试验中事件发生的次数,则:
    \[P(X = k) = \frac{\lambda ^k}{k!}e^{-\lambda },\quad k = 0,1,\cdots,n  \]
    称X服从参数为n的泊松分布,记为$X~P(\lambda ) $

    note:1)泊松逼近定理:设X~B(n,p)$\lim_{n \to \infty}np_n =\lambda  $  
    \[lim_{n \to \infty}C_n^kp^k(1-p)^{n-k} = \frac{\lambda ^k}{k!}e^{-\lambda } ,\quad k = 0,1,\cdots,\] 
    说明泊松分布就是特殊情况下的二项分布

    3.几何分布:将样本空间$\varOmega  $以$A,\overline{A} $做划分,$P(A) = p,1 - P(A) = q $
    设离散型随机变量X为在n次独立重复的试验中事件A首次发生时的试验次数,则:
    \[P(X = k) = pq^{k-1},\quad k = 1,2,\cdots, \]
    称X服从参数为n的几何分布,记为$X~G(p) $

    note:无记忆性:若$X~G(p) $,则
    \[P(X > n+ t|X > n) = P(X > t) \]

    4.超几何分布:取出问题,无放回,无次序
    \[P(X = k) = \frac{C_M^kC_{N-M}^{N-k}}{C_N^n} \]
    当N趋于无穷大时,类似于二项分布
    \subsection{连续型随机变量}
    在非离散的随机变量中,仅讨论分布函数可以写为
    $F(x) = \int_{x}^{-\infty}f(t)  \,dt  $的连续型随机变量
    \subsubsection{连续性随机变量的定义与密度函数}
    以几何概型为例,在区间上随机放点,放在某一个区域的概率$P(c\quad d) = \frac{c - d}{b - a} $
    那么对于在该区间上的一个点e,$P(e\quad e) = 0 $,无法运算,所以我们使用概率密度来描述

    概率密度为单位长度的概率,在求某一个区间的概率时,只需要对长度积分即可
    
    1.定义:对于一个分布函数F(x)满足
    \[\int_{-\infty}^{x}f(t)  \,dt  \]
    则X是连续性随变量, f(t)是它的概率密度函数

    2.性质:1)非负性。2)归一性

    note:1)连续型随机变量的分布函数连续,但是分布函数连续不能保证随机变量是连续型

    2)在密度函数f(x)的连续点$x_0$处,有$F'(x_0) = f(x_0)$

    3)
    \[P(a < X \leq b) = \int_{a}^{b}f(x) \,dx\]
    \subsubsection{常见的连续型随机变量}
    1.均匀分布

    在一个区间内,等可能任取一点记为X,X的密度函数:
    \[f(x) = 
    \begin{cases}
        \frac{1}{b - a} & \text{if } a<x<b \\
        0 & \text{if } \text{其他}
    \end{cases} \]
    称X在区间(a,b)上服从均匀分布,记为X~U(a,b)
    做积分的均匀分布的分布函数
    \[F(x) = 
    \begin{cases}
        0 & \text{if } x<a \\
        \frac{x-a}{b-a} & \text{if } a\leq x<b \\
        1 & \text{if } b\leq x
    \end{cases}\]
    2.指数分布

    电子产品的工作寿命X,则X的密度函数:
    \[f(x) = 
    \begin{cases}
        \lambda e^{-\lambda x} & , x > 0 \\
        0 & , x \leq 0
    \end{cases}\]
    其中参数为$\lambda $,称X服从参数为$\lambda $的指数分布,记作$X~E(\lambda )$

    分布函数:
    \[F(x) = 
    \begin{cases}
        1-e^{-\lambda x} & ,x > 0 \\
        0 & , x \leq 0
    \end{cases}\]
    类似于几何分布,指数分布也有无记忆性:

    \[P(X>t+s|X>t) = P(X>s)\]
    3.正态分布

    对于一种统计数据X,将其当作随机变量,则概率密度:
    \[f(x) = \frac{1}{\sqrt{2\pi }\sigma }e^{-\frac{(x-\mu )^2}{2\sigma ^2}}, x \in R\]
    其中$\mu \in R,\sigma >0$,称X服从正态分布,记作$X~N(\mu ,\sigma ^2)$

    $\mu $位置参数,f(x)关于$x=\mu $f对称,$\sigma $形状参数,$\sigma $越小f(x)图像越窄越高

    正态分布的分布函数:
    \[F(x) = \frac{1}{\sqrt{2\pi }\sigma}\int_{-\infty}^{x}e^{-\frac{(t-\mu)^2}{2\sigma^2}}  \,dt \]
    标准正态分布:取$\mu = 0 ,\sigma  = 1$ 密度函数:
    \[\phi (x) = \frac{1}{\sqrt{2\pi}}e^{-\frac{x^2}{2}},x\in R\]
    分布函数:
    \[\Phi (x) = \frac{1}{\sqrt{2\pi}}\int_{-\infty}^{x}e^{-\frac{t^2}{2}}\,dt \]
    note:正态分布化标准正态分布:若$X~N(\mu ,\sigma ^2)$,则$\frac{X-\mu}{\sigma} ~N(0,1)$
    \subsection{随机变量函数的分布}
    已知随机变量X和其密度函数分布函数,而$Y = g(x)$,一般情况,Y是和X同类型的随机变量,如何求出Y的密度函数
    \subsubsection{离散型}
    已知$P(X = x_i) = p_i$,则$P(Y = g(x_i)) = p_i$

    当函数Y=g(X)不单调时,得到的相同的Y概率要求和
    \subsubsection{连续型}
    步骤:
    \begin{minipage}[t]{0.9\linewidth}
        1.由X的取值范围求出Y的取值范围

        2.求出Y的分布函数$F_Y(y)$

        3.对Y的分布函数计算积分后求导(均匀分布)或者直接求导得到密度函数
        
    \end{minipage}
    note:当y = g(x)是严格单调函数,并且具有一阶连续导数时,x = h(y)是y = g(x)的反函数,则Y = g(X)的密度函数:
    \[f_Y(y) = f_X(h(y))|h'(y)|\]


    
\end{document}