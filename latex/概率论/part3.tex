\documentclass[11pt,twoside,a4paper]{ctexart}
\usepackage[backend=bibtex]{biblatex}
\usepackage[margin=1in]{geometry}
\usepackage{amsmath}
\usepackage{amssymb}
\begin{document}
    \section{二维随机变量及其分布}
    \subsection{二维随机变量的联合分布和边际分布}
    \subsubsection{二维随机变量的联合分布函数与性质}
    1.联合分布函数:
    \[F(x,y) = P(X\leq x,Y\leq y)\]

    2.性质:
    \begin{minipage}[t]{0.9\linewidth}
        1)F(x,y)分别关于x,y单调不减:
        \[x_1<x_2  \Rightarrow  F(x_1,y) \leq F(x_2,y) \]
        \[y_1<y_2  \Rightarrow  F(x,y_1) \leq F(x,y_2)  \]

        2)F(x,y)满足非负有界性:
        \[0\leq F(x,y)\leq 1;F(+\infty,+\infty) = 1\]
        对任意x,y:$F(x,-\infty) = F(-\infty,y) = F(-\infty,-\infty) = 0$

        显然$F(x,+\infty) = F_X(x)$

        3)F(x,y)分别关于x,y右连续:$F(x+0,y) = F(x,y),F(x,y+0) = F(x,y) $

        4)若$x_1<x_2,y_1<y_2$,则
        \[P(x_1<X<x_2,y_1<Y<y_2) = F(x_2,y_2) - F(x_1,y_2) - F(x_2,y_1) + F(x_1,y_1)\]
    \end{minipage}

    \subsubsection{边际分布函数}
    F(x,y)是(X,Y)的联合分布函数,在二维变量里X的分布函数为:
    \[F_X(x) = P(X\leq x) =P(X\leq x,Y\leq +\infty) = F(x,+\infty ); x\in R\]
    称为边际分布函数,$F_Y(y)$同理

    一般从联合分布函数可以求出边际分布函数,但是由两个边际分布函数很难求出联合分布函数、

    \subsection{二维离散随机变量}
    当X,Y都是离散型随机变量时
    
    1.联合分布列:$p_{ij} = P(X = x_i,Y = y_j)$

    2.联合分布列的性质:1)非负性;2)归一性

    note:由联合分布列可以求出联合分布函数:$F(x,y) = \sum _{x_i\leq x}\sum _{y_i\leq y}p_{ij}$
    \subsubsection{离散型随机变量的边际分布}
    1.X的边际分布$p_{i\cdot}$:$P(X = x_i) = P(X = x_i,Y < +\infty),p_{i\cdot} = \sum _j  p_{ij}$,Y同理

    Y的边际分布$p_{\cdot j}$同理

    \subsubsection{二维离散随机变量的独立性}
    1.定义:(X,Y)是二维随机变量,对于X,Y所有的取值$x_1,X_2,\cdots y_1,y_2,\cdots$若有
    \[P(X = x_i,Y = y_j) = P(X = x_i)P(Y = y_j)\]
    则称X,Y相互独立

    \subsubsection{二维离散型随机变量的条件分布列}
    1.定义:(X,Y)是二维随机变量,对于X,Y所有的取值$x_1,X_2,\cdots y_1,y_2,\cdots$称
    \[p_{i|j} \triangleq P(X = x_i|Y = y_j) = \frac{P(X = x_i,Y = y_j)}{P(Y = y_j)}\]
    为已知${Y = y_i}$条件下的X的分布列
    
    对于Y同理可以定义

    \subsection{二维连续型随机变量及其分布}
    \subsubsection{联合分布密度}
    1.定义:F(x,y)分布函数,存在非负函数f(x,y)使得所有的实数x,y有:
    \[F(x,y) = P(X\leq x,Y\leq y) = \int_{-\infty}^x\int_{-\infty}^yf(u,v)\,dvdu\]
    
    2.f(x,y)性质:
    \begin{minipage}[t]{0.9\linewidth}
        1)$f(x,y)\geq 0$

        2)$\int_{-\infty}^{+\infty}\int_{-\infty}^{+\infty}f(x,y)\,dxdy = 1$

        3)F(x,y)是二元连续函数

        4)在f(x,y)的连续点处有
        \[\frac{\partial ^2F(x,y)}{\partial x \partial y} = f(x,y)\]

    \end{minipage}

        note:X,Y一维连续,不一定有(X,Y)二维连续;但(X,Y)二维连续,一定有X,Y一维连续
    \subsubsection{二维连续型随机变量的边际密度}
    1.定义:(X,Y)是二维连续型随机变量,联合密度函数f(x,y),X的边际密度函数:
    \[f_X(x) = \int _{-\infty}^{+\infty}f(x,y)\,dy, x\in \mathbb{R}\]
    Y的边际密度函数同理定义

    \subsubsection{常用二维连续型随机变量的分布}
    1.二维均匀分布
    设D为平面有界闭区间,其面积$S_D$,若密度函数:
    \[f(x,y) = 
    \begin{cases}
        \frac{1}{S_D} &, (x,y)\in D \\
        0 &, (x,y)\notin D
    \end{cases}\]
    称为二维变量(X,Y)服从D上的二维均匀分布

    若G是D的子区域,则
    \[P((x,y)\in G) = \iint _G f(x,y) \, d\sigma = \frac{1}{S_D}\iint _G \,d\sigma = \frac{S_G}{S_D}\]
    2.二维正态分布

    记忆:$(X,Y)\text{~}N(\mu _1,\mu _2,\sigma _1,\sigma _2,\rho )$,则$X\text{~}N(\mu _1,\sigma _1^2),Y\text{~}N(\mu _2,\sigma _2^2)$

    \subsubsection{二维连续型随机变量的独立性}
    1.定义:$f(x,y) = f_X(x)f_Y(y)$
    或$F(x,y) = F_X(x)F_Y(y)$

    2.性质:$(X,Y)~N(\mu _1,\mu _2,\sigma _1,\sigma _2,\rho )$,则XY相互独立的充要条件$\rho = 0$
    \subsubsection{二维连续型随机变量的条件密度}
    由于二维连续型随机变量在某一条线上的概率是0,所以无法用和离散型随机变量一样的公式处理,选择使用分布函数代替

    1.条件分布函数:1)极限求 p78;2)$F_{X|Y}(x|y) = \int _{-\infty}^{x}\frac{f(u,y)}{f_Y(y)}\,du$

    2.条件密度函数:$f_{X|Y}(x|y) = \frac{f(x,y)}{f_Y(y)}  (f_Y(y) > 0)$
\end{document}