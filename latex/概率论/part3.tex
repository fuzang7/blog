\documentclass[11pt,twoside,a4paper]{ctexart}
\usepackage[backend=bibtex]{biblatex}
\usepackage[margin=1in]{geometry}
\usepackage{amsmath}
\usepackage{amssymb}
\begin{document}
    \section{二维随机变量及其分布}
    \subsection{二维随机变量的联合分布和边际分布}
    \subsubsection{二维随机变量的联合分布函数与性质}
    1.联合分布函数:
    \[F(x,y) = P(X<x,Y<y)\]

    2.性质:
    \begin{minipage}[t]{0.9\linewidth}
        1)F(x,y)分别关于x,y单调不减:
        \[x_1<x_2  \Rightarrow  F(x_1,y) \leq F(x_2,y) \]
        \[y_1<y_2  \Rightarrow  F(x,y_1) \leq F(x,y_2)  \]

        2)F(x,y)满足非负有界性:
        \[0\leq F(x,y)\leq 1;F(+\infty,+\infty) = 1\]
        对任意x,y:$F(x,-\infty) = F(-\infty,y) = F(-\infty,-\infty) = 0$

        显然$F(x,+\infty) = F_X(x)$

        3)F(x,y)分别关于x,y右连续:$F(x+0,y) = F(x,y),F(x,y+0) = F(x,y) $

        4)若$x_1<x_2,y_1<y_2$,则
        \[P(x_1<X<x_2,y_1<Y<y_2) = F(x_2,y_2) - F(x_1,y_2) - F(x_2,y_1) + F(x_1,y_1)\]
    \end{minipage}

    \subsubsection{边际分布函数}
    F(x,y)是(X,Y)的联合分布函数,在二维变量里X的分布函数为:
    \[F_X(x) = P(X\leq x) =P(X\leq x,Y\leq +\infty); x\in R\]
    称为边际分布函数,$F_Y(y)$同理

    \subsection{二维离散随机变量}
    当X,Y都是离散型随机变量时
    
    1.联合分布列:$p_{ij} = P(X = x_i,Y = y_j)$

    2.联合分布列的性质:1)非负性;2)归一性

    note:由联合分布列可以求出联合分布函数:$F(x,y) = \sum _{x_i\leq x}\sum _{y_i\leq y}p_{ij}$
    \subsubsection{离散型随机变量的边际分布}
    1.X的边际分布:$P(X = x_i) = P(X = x_i,Y < +\infty),p_{i\cdot} = \sum _j = p_{ij}$,Y同理

    \subsubsection{条件分布函数}

    \subsection{二维连续型随机变量及其分布}
    \subsubsection{联合分布密度}
    1.定义:F(x,y)分布函函数,存在非负函数f(x,y)使得所有的实数x,y有:
    \[F(x,y) = P(X\leq x,Y\leq y) = \int_{-\infty}^x\int_{-\infty}^yf(u,v)\,dvdu\]
    
    2.f(x,y)性质:
    \begin{minipage}[t]{0.9\linewidth}
        1)$f(x,y)\geq 0$

        2)$\int_{-\infty}^{+\infty}\int_{-\infty}^{+\infty}f(x,y)\,dxdy = 1$

        3)

        4)

        5)
    \end{minipage}

\end{document}