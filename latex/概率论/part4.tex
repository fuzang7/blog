\documentclass[11pt,twoside,a4paper]{ctexart}
\usepackage[backend=bibtex]{biblatex}
\usepackage[margin=1in]{geometry}
\usepackage{amsmath}

\begin{document}
    \section{随机变量的数字特征}
    \subsection{随机变量的数学期望}
    \subsubsection{离散型}
    定义:若级数$\sum _{k=1}^{\infty}x_kp_k$绝对收敛,则
    \[EX = \sum _{k=1}^{+\infty}x_kp_k\]

    note:重点注意此处无限时的情况,对于有限个离散型随机变量,显然必然存在其期望,但是当有无限个随机变量时就要注意存在性

    \subsubsection{连续型}
    定义:若积分$\int_{-\infty}^{+\infty}|x|f(x)  \,dx $收敛,则
    \[E(X) = \int_{-\infty}^{+\infty}xf(x)  \,dx \]

    \subsubsection{函数}
    设Y是随机变量X的函数$Y = g(X)$g(x)是连续函数
    1)X是离散型随机变量
    \[EY = \sum _{k=1}^{\infty}g(x_k)p_k\]
    2)X是连续型随机变量
    \[E(Y) = \int_{-\infty}^{+\infty}g(x)f(x)  \,dx \]

    \subsubsection{二维变量}
    设Z是随机变量X,Y的函数,Z = g(X,Y),g是连续函数
    1)(X,Y)是二维离散型随机变量
    \[EZ =\sum _{i=1}^{\infty}\sum _{j=1}^{\infty}g(x_i,y_j)p_{ij}\]
    \[EX =\sum _{i=1}^{\infty}\sum _{j=1}^{\infty}x_ip_{ij}\]
    \[EY =\sum _{i=1}^{\infty}\sum _{j=1}^{\infty}y_jp_{ij}\]
    2)(X,Y)是二维连续型随机变量
    \[E(Z) =\int_{-\infty}^{+\infty}\int_{-\infty}^{+\infty}g(x,y)f(x,y) \,dxdy\]
    \[E(X) =\int_{-\infty}^{+\infty}\int_{-\infty}^{+\infty}xf(x,y) \,dxdy \]
    \[E(Y) =\int_{-\infty}^{+\infty}\int_{-\infty}^{+\infty}yf(x,y) \,dxdy\]

    \subsubsection{期望的性质}
    1.$E(c) = c$常数的期望是本身,自身没有变化,期望就展示自己

    2.$E(cX) = cE(X)$期望内的系数可以提出来,线性

    3.$E(X_1 + X_2) = E(X_1) + E(X_2)$两个期望相加减依然满足线性

    4.若$X_1,X_2$互相独立,则$E(X_1X_2) = E(X_1)E(X_2)$

    note:1)综合使用123,$E(aX_1 + b + cX_2 + d) = aE(X_1) + b + cE(X_2) + d$,记忆期望满足线性相加

    2)由两个互相独立的随机变量的函数也互相独立得到:$E(g(X_1)f(X_2))=E(f(X_2))E(g(X_1))$

    \subsection{方差}
    \subsubsection{随机变量的方差}
    1.定义:对于随机变量X,方差$DX = E\{[X - E(X)]^2\}$,标准差$\sigma _x = \sqrt{D(X)}$

    2.计算时,方差一般用$D(X) = E(X^2) - [E(X)]^2$计算时

    \subsubsection{方差的性质}
    1.$D(c) = 0$常数不偏离其期望

    2.$D(aX + b) = a^2D(X)$,定义的平方造成了方差的非线性

    3.若X,Y相互独立,$D(X + Y) = D(X) + D(Y)$

    \subsubsection{常见分布的期望与方差}
    1.0-1分布,X\textasciitilde B(1,p),$EX = p,DX = p(1-p)$

    2.二项分布,X\textasciitilde B(n,p)$EX = np,DX = np(1-p)$

    3.泊松分布,X\textasciitilde P($\lambda $) ,$EX = \lambda ,dx = \lambda $

    4.几何分布,X\textasciitilde G(p),$EX = \frac{1}{p},DX = \frac{q}{p^2}$

    5.均匀分布,X\textasciitilde U(a,b),$EX = \frac{a+b}{2}, DX = \frac{(b-a)^2}{12}$

    6.指数分布,X\textasciitilde E($\lambda $),$E(X) = \frac{1}{\lambda},D(X) = \frac{1}{\lambda ^2}$

    7.正态分布,X\textasciitilde N($\mu , \sigma ^2$),$E(X) = \mu , D(X) = \sigma ^2$

    \subsection{协方差和相关系数}

    \subsubsection{协方差}

    1.定义:二维随机变量(X,Y),$Cov(X,Y) = E\{[X - E(X)][Y - E(Y)]\} = E(XY) - E(X)E(Y)$

    2.性质:
    \begin{minipage}[t]{0.9\linewidth}
        1)$Cov(X,Y) = Cov(Y,X)$

        2)$Cov(X,X) = D(X)$

        3)$Cov(aX,bY) = abCov(X,Y)$

        4)若X,Y相互独立,则$Cov(X,Y) = 0$

        5)$Cov(X_1 + X_2 , Y) = Cov(X_1 , Y) + Cov(X_2 , Y)$

        6)$D(aX + bY) = a^2D(X) + b^2D(Y) + 2abCov(X,Y)$
        
    \end{minipage}

    \subsubsection{相关系数}

    1.定义:$\rho _{XY} = \frac{Cov(X,Y)}{\sqrt{DX} \sqrt{DY}}$

    2.性质:1)$|\rho _{XY}| \leq 1 $;2)$|\rho _{XY}| = 1$时,存在$Y = aX + b$

    note:相关系数仅描述了线性的相关程度,是一种弱于独立的条件,独立则必然不相关,但是不相关不一定独立

    3.二维正态随机变量$(X,Y)\sim N(\mu _1,\mu_2,\sigma _1,\sigma _2,\rho)$,XY的相关系数是$\rho $

    4.若(X,Y)服从二维正态,则X,Y互相独立等价于X,Y互不相关

    5.二维正态分布(X,Y),令$Z = aX + bY,W = cX + dY$则(Z,W)也服从二维正态分布

    \subsubsection{随机变量的标准化}
    
    1.定义:随机变量X的期望与方差都存在则
    \[X' = \frac{X - EX}{\sqrt{DX}}\]
    为X的标准化随机变量

    2.性质:1)$EX' = 0,DX' = 1$;2)$\rho _{X,Y} = \rho _{X',Y'} = EX'Y'$

    \subsection{其他数字特征}

    \subsubsection{矩}
    1.定义:X为随机变量,如果$E|X|^k < +\infty $,则称$EX^k$为随机变量的k阶原点矩,$E(X - EX)^k$为X的k阶中心矩

    \subsubsection{协方差矩阵}
    1.定义:设有n个随机变量,令$\sigma _{ij} = Cov(X_i,X_j)$,则矩阵$\sum = (\sigma _{ij})_{n*n}$为这n个随机变量的协方差矩阵

    2.性质:1)协方差矩阵是对称矩阵;2)协方差矩阵是半正定矩阵
\end{document}