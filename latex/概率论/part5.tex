\documentclass[11pt,twoside,a4paper]{ctexart}
\usepackage[backend=bibtex]{biblatex}
\usepackage[margin=1in]{geometry}
\usepackage{amsmath}
\begin{document}
\section{大数定理与中心极限定理}
\subsection{大数定理}
\subsubsection{切比雪夫不定式} 
1.定理:随机变量X,$EX = \mu  , DX = \sigma ^2 $对任意正数$\varepsilon $ 有
\[P\{|X-\mu|\geq \varepsilon \}\leq \frac{\sigma ^2}{\varepsilon ^2} \]
也可写为
\[P\{|X-\mu| < \varepsilon \}\geq 1 - \frac{\sigma ^2}{\varepsilon ^2}\]  

\subsubsection{大数定律}
1.定义:对于一个随机变量序列,对于任意正数$\varepsilon $有
\[\lim_{n \to +\infty}P\{|X_n - a|\geq \varepsilon\} = 0\]
或
\[\lim_{n \to +\infty}P\{|X_n - a| < \varepsilon\} = 1\]
称为随机变量序列依概率收敛于a

note:这里的极限是在概率意义下的,表示当n充分大时序列发生的概率越来越接近于a

2.切比雪夫大数定律

设随机变量序列相互独立,每一个随机变量存在有限的方差,且一致有界,则对于任意正数$\varepsilon $ 有
\[ \lim_{n \to +\infty}P\{  | \frac{1}{n}\sum _{i=1}^n X_i - \frac{1}{n}\sum _{i=1}^n E(X_i) | \geq \varepsilon \} = 0 \]
表示n很大时,随机变量的算数平均值概率意义上接近其数学期望

3.伯努利大数定律

设$n_A$是在n重伯努利试验中事件A发生的概率,则对任意正数$\varepsilon $ 有
\[\lim_{n \to +\infty}P\{|\frac{n_A}{n} - p|\geq \varepsilon  \} = 0 \]
表示当n充分大时事件A发生的频率概率意义上接近事件A的概率

4.辛钦大数定律

设随机变量序列独立同分布,则对于任意正数$\varepsilon $ 有
\[\lim_{n \to +\infty}P\{|\frac{1}{n}\sum _{i=1}^n X_i -\mu | > \varepsilon \} = 0\]
表示$\frac{1}{n}\sum _{i=1}^n X_i$概率收敛于$\mu $

\subsection{中心极限定理}

1.独立同分布的中心极限定理

设随机变量序列相互独立同分布,则
\[\lim_{n \to +\infty}P\{\frac{\sum _{i=1}^n X_i - n\mu }{\sqrt n \sigma }\leq x \} = \Phi (x)\]
表示独立同分布的随机变量序列的和的标准化是正态分布

2.拉普拉斯中心极限定理

设随机变量X表示n重伯努利试验中事件A发生的次数,则
\[\lim_{n \to +\infty}P\{\frac{X - np}{\sqrt {np(1-p)}} \leq x \} = \frac{1}{\sqrt{2\pi }} \int_{-\infty }^{x}e^{-\frac{t^2}{2}}  \,dt  = \Phi (x)\]
表示正态分布是二项分布的极限分布,n足够大时,二项分布x的标准化服从标准正态


\end{document}