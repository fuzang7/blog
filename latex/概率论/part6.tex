\documentclass[11pt,twoside,a4paper]{ctexart}
\usepackage[backend=bibtex]{biblatex}
\usepackage[margin=1in]{geometry}
\usepackage{amsmath}
\begin{document}
\section{数理统计的基本概念}
\subsection{总体,样本,统计量}
总体:研究对象全体,可以用一个随机变量X及其分布F表示

个体:总体中的每一个成员

抽样:按照一定法则,从总体中抽取部分个体观测的过程,常用有不放回抽样和放回抽样,在样本容量相对总体过小时这两个方法可以视为等同

样本:抽样选中的部分个体,用n个随机变量来表示

样本容量:样本中个体的数量

样本观测值:由样本中的n个随机变量观察得到的n个数据

简单随机抽样满足:1)代表性:总体中每个个体都有相同的被抽到的概率,则样本中n个随机变量和总体X有相同的分布 

\  2)独立性:n个随机变量相互独立

简单随机样本:由简单随机抽样得到的样本

统计量:设$X_1,X_2,\cdots ,X_n$是总体X的一个样本,若$T = T(X_1,X_2,\cdots ,X_n)$是样本的函数,且不含任意未知参数,则称T是一个统计量

常见统计量:
\begin{minipage}[t]{0.9\linewidth}
    样本均值:
    \[\overline{X} = \frac{1}{n}\sum _{i=1}^n X_i\]

    样本方差:
    \[S^2 = \frac{1}{n-1}\sum _{i=1}^n (X_i - \overline{X})^2\]

    样本标准差:
    \[S = \sqrt{\frac{1}{n-1}\sum _{i=1}^n (X_i - \overline{X})^2} \]

    样本k阶原点矩:
    \[A_k = \frac{1}{n}\sum _{i=1}^n X_i^k\]

    极大次序统计量:
    \[X_{(n)} = max\{X_1,X_2,\cdots ,X_n \}\]

    极小次序统计量:
    \[X_{(1)} = min\{X_1,X_2,\cdots ,X_n \}\]

\end{minipage}

\subsection{常用统计量的分布}

\subsubsection{标准正态}
正态分布的可加性:对于相互独立的随机变量$X_1,X_2,\cdots , X_n$且$X_i \sim N(\mu _i,\sigma _i^2)$
\[\sum _{i=1}^na_iX_i + c \sim N(\sum _{i=1}^na_i \mu _i + c,\sum _{i=1}^n a_i^2\sigma _i^2)\]

\subsubsection{$\chi ^2$分布}
1.$\chi ^2$分布:设$X_1,X_2,\cdots ,X_n$相互独立且服从标准正态分布,则
\[\chi ^2 = X_1^2 + X_2^2 + \cdots + X_n^2\]
服从自由度为n的$\chi ^2$分布,记为$\chi^2(n)$,其密度函数为:
\[f(x) = 
\begin{cases}
    \frac{1}{2^(\frac{n}{2}) \Gamma (\frac{n}{2})}e^{-\frac{x}{2}}x^{\frac{n}{2}-1} & x>0, \\
    0 & x \leq 0
\end{cases}\]

2.性质
1)可加性:若X$\sim  \chi^2(n)$,Y$ \sim \chi^2(m)$,则$X + Y \sim \chi^2(n+m)$

2)$E[\chi ^2(n)] = n,D[\chi ^2(n)] = 2n$

\subsubsection{t分布}
1.设X$\sim $N(0,1),Y$\sim \chi^2(n)$,且X和Y相互独立,则
\[t = \frac{X}{\sqrt{Y/n}}\]
服从自由度为n的t分布,记为t(n),其密度函数为
\[f(X) = \frac{\Gamma (\frac{n+1}{2})}{\sqrt{n\pi }\Gamma (\frac{n}{2})}(1 + \frac{x^2}{n}^{-\frac{n+1}{2}})\]
当n充分大时,t分布可以看作标准正态

\subsubsection{F分布}
1.$X\sim \chi^2(n),Y\sim \chi^2(m)$,且X,Y相互独立,则
\[F = \frac{X/n}{Y/m}\]
服从自由度为n,m的F分布,记为F(n,m)

2.性质:
1)$X\sim F(n,m)$,则$\frac{1}{X}\sim F(m,n)$

2)若$t\sim t(n)$,则$t^2 \sim F(1,n)$

\subsection{正态总体的抽样分布}
1.目的:总体为正态分布,进行抽样,求出抽样得到的样本的统计量服从什么分布

\subsubsection{单正态总体的抽样分布定理}
设总体$X\sim N(\mu ,\sigma ^2)$,简单随机样本$X_1,X_2,\cdots ,X_n$,样本均值$\overline{X} = \frac{1}{n}\sum _{i=1}^n X_i$,
样本方差$S^2 = \frac{1}{n-1}\sum _{i=1}^n(X_i - \overline{X})^2$,则

1)$\frac{\overline{X} - \mu }{\sigma / \sqrt {n}} \sim N(0,1)$,已知方差

2)$\frac{n-1}{\sigma ^2}S^2 \sim \chi ^2(n-1)$,且$\overline{X},S^2$相互独立

3)$\frac{\overline{X} - \mu }{S/\sqrt{n}} \sim t(n-1)$,未知方差

\subsubsection{双正态总体的抽样分布定理}
设总体$X\sim N(\mu _1,\sigma_1 ^2),Y\sim N(\mu _2,\sigma _2 ^2)$相互独立,简单随机样本$X_1,X_2,\cdots ,X_n;Y_1,Y_2,\cdots ,Y_m$,则有

1)$\frac{(\overline{X} - \overline{Y}) - (\mu _1 - \mu _2)}{\sqrt{\frac{\sigma _1^2}{n} + \frac{\sigma _2^2}{m}}} \sim N(0,1)$;$\sigma $已知

2)$\frac{S^2_1/S^2_2}{\sigma _1^2/\sigma _2^2} \sim F(n - 1,m - 1)$

3)若$\sigma _1^2 = \sigma _2^2$,但未知大小
\[\frac{(\overline{X} - \overline{Y}) - (\mu _1 - \mu _2)}{S_\omega \sqrt{\frac{1}{n} + \frac{1}{m}}} \sim t(n + m - 2)\]
其中
\[S_\omega = \sqrt{\frac{(n - 1)S_1^2 + (m - 1)S_2^2}{n + m - 2}}\]

\subsection{抽样分布的上$\alpha $分位点}

\subsubsection{标准正态}

设随机变量$Z\sim N(0,1)$,对$\alpha \in (0,1)$,实数$z_\alpha$ 满足
\[P\{Z > z_\alpha\} = \alpha\]
则称$z_\alpha $是标准正态上的$\alpha$分位点

note:1)$\alpha$是概率,$z_\alpha$是对应x的取值;2)$z_{1 - \alpha} = - z_\alpha$

\subsubsection{$\chi ^2$分布}
设随机变量$\chi ^2\sim \chi^2(n)$,对$\alpha \in (0,1)$,实数$\chi ^2_\alpha(n)$ 满足
\[P\{\chi ^2 > \chi ^2_\alpha(n)\} = \alpha\]
则称$\chi ^2_\alpha(n) $是标准正态上的$\alpha$分位点
\[P\{\chi ^2(n) \leq \chi ^2_{1 - \alpha}(n)\} = \alpha\]

note:$\chi^2$分布图像不对称

\subsubsection{t分布}

设随机变量$t\sim t(n)$,对$\alpha \in (0,1)$,实数$t_\alpha(n)$ 满足
\[P\{t > t_\alpha(n)\} = \alpha\]
则称$t_\alpha(n) $是标准正态上的$\alpha$分位点

note:$t_{1 - \alpha} = -t_\alpha $

\subsubsection{F分布}

设随机变量$F\sim F(n,m)$,对$\alpha \in (0,1)$,实数$F_\alpha(n,m)$ 满足
\[P\{F > F_\alpha(n,m)\} = \alpha\]
则称$F_\alpha(n,m) $是标准正态上的$\alpha$分位点

note:F分布图像不对称但是$F_{1 - \alpha}(n,m) = \frac{1}{F_\alpha (m,n)}$,注意参数位置的调整

\subsubsection{一般情况}
设连续型随机变量Y,对$\alpha \in (0,1)$,实数$Y_\alpha$ 满足
\[P\{Y > Y_\alpha\} = \alpha\]
则称$t_\alpha(n) $是标准正态上的$\alpha$分位点

易得:
\begin{minipage}[t]{0.9\linewidth}

    1)$P\{Y > Y_\alpha\} = \alpha $

    2)$P\{Y < Y_{1 - \alpha}\} = \alpha$

    3)$P\{Y < Y_{1 - \alpha/2} \text{或} Y > Y_{\alpha/2}\} = \alpha$

    4)$P\{Y < Y_ \alpha\} =1 - \alpha$

    5)$P\{Y > Y_{1 - \alpha}\} = 1 - \alpha$

    6)$P\{Y_{1 - \alpha /2} < Y < Y_{\alpha/2}\} = 1 - \alpha$
    
\end{minipage}


\end{document}