\documentclass[11pt,twoside,a4paper]{ctexart}
\usepackage[backend=bibtex]{biblatex}
\usepackage[margin=1in]{geometry}
\usepackage{amsmath}
\begin{document}
\section{数理统计的基本概念}
\subsection{总体,样本,统计量}
总体:研究对象全体,可以用一个随机变量X及其分布F表示

个体:总体中的每一个成员

抽样:按照一定法则,从总体中抽取部分个体观测的过程,常用有不放回抽样和放回抽样,在样本容量相对总体过小时这两个方法可以视为等同

样本:抽样选中的部分个体,用n个随机变量来表示

样本容量:样本中个体的数量

样本观测值:由样本中的n个随机变量观察得到的n个数据

简单随机抽样满足:1)代表性:总体中每个个体都有相同的被抽到的概率,则样本中n个随机变量和总体X有相同的分布 

\  2)独立性:n个随机变量相互独立

简单随机样本:由简单随机抽样得到的样本

统计量:设$X_1,X_2,\cdots ,X_n$是总体X的一个样本,若$T = T(X_1,X_2,\cdots ,X_n)$是样本的函数,且不含任意未知参数,则称T是一个统计量

常见统计量:
\begin{minipage}[t]{0.9\linewidth}
    样本均值:
    \[\overline{X} = \frac{1}{n}\sum _{i=1}^n X_i\]

    样本方差:
    \[S^2 = \frac{1}{n-1}\sum _{i=1}^n (X_i - \overline{X})^2\]

    样本标准差:
    \[S = \sqrt{\frac{1}{n-1}\sum _{i=1}^n (X_i - \overline{X})^2} \]

    样本k阶原点矩:
    \[A_k = \frac{1}{n}\sum _{i=1}^n X_i^k\]

    极大次序统计量:
    \[X_{(n)} = max\{X_1,X_2,\cdots ,X_n \}\]

    极小次序统计量:
    \[X_{(1)} = min\{X_1,X_2,\cdots ,X_n \}\]

\end{minipage}

\subsection{常用统计量的分布}

\subsubsection{标准正态}
\subsubsection{$\chi ^2$分布}
1.$\chi ^2$分布:设$X_1,X_2,\cdots ,X_n$相互独立且服从标准正态分布,则
\[\chi ^2 = X_1^2 + X_2^2 + \cdots + X_n^2\]
服从自由度为n的$\chi ^2$分布,记为$\chi^2(n)$,其密度函数为:
\[f(x) = 
\begin{cases}
    \frac{1}{2^(\frac{n}{2}) \Gamma (\frac{n}{2})}e^{-\frac{x}{2}}x^{\frac{n}{2}-1} & x>0, \\
    0 & x \leq 0
\end{cases}\]

2.性质
1)可加性:若X$\sim  \chi^2(n)$,Y$ \sim \chi^2(m)$,则$X + Y \sim \chi^2(n+m)$

2)$E[\chi ^2(n)] = n,D[\chi ^2(n)] = 2n$

\subsubsection{t分布}
1.设X$\sim $N(0,1),Y$\sim \chi^2(n)$,且X和Y相互独立,则
\[t = \frac{X}{\sqrt{Y/n}}\]
服从自由度为n的t分布,记为t(n),其密度函数为
\[f(X) = \frac{\Gamma (\frac{n+1}{2})}{\sqrt{n\pi }\Gamma (\frac{n}{2})}(1 + \frac{x^2}{n}^{-\frac{n+1}{2}})\]
当n充分大时,t分布可以看作标准正态

\subsubsection{F分布}
1.$X\sim \chi^2(n),Y\sim \chi^2(m)$,且X,Y相互独立,则
\[F = \frac{X/n}{Y/m}\]
服从自由度为n,m的F分布,记为F(n,m)

2.性质:
1)$X\sim F(n,m)$,则$\frac{1}{X}\sim F(m,n)$

2)若$t\sim t(n)$,则$t^2 \sim F(1,n)$


\end{document}