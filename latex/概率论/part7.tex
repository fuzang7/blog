\documentclass[11pt,twoside,a4paper]{ctexart}
\usepackage[backend=bibtex]{biblatex}
\usepackage[margin=1in]{geometry}
\usepackage{amsmath}
\begin{document}
    
\section{参数的点估计以及优越性}
\subsection{点估计}
点估计:通过样本值求出总体参数的一个具体估计量或估计值
\subsubsection{矩估计法}
1.思路:用样本矩替换同阶总体矩

2.过程:
\begin{minipage}[t]{0.9\linewidth}
    1)令$EX = \overline{X}$

    2)代入参数得到$h(\theta ) = \overline{X}$

    3)将样本求出的$\overline{X}$代入上式,解出未知量$\theta $
    
\end{minipage}

在存在多个未知数的情况时,由大数定律
\[E(X^i) = A_i = \frac{1}{n}\sum _{j=1}^n x_j^i\]
列出方程组求解即可

\subsubsection{最大似然估计法}
1.思路:设一个随机试验由若干种可能$A_1,A_2,\cdots $,在一次试验中出现了结果$A_k$,认为试验$A_k$出现概率大,将事件$A_k$的概率写为
$P_{A_k} = h(\theta )$,令$P_{A_k}$最大,求出对应的$\theta $

2.似然函数:设$X_1,X_2,\cdots , X_n$时简单随机样本,$x_1,x_2,\cdots ,x_n$为样本观测值,称
\[L(\theta ) = \prod _{i=1}^n p(x_i,\theta )\]
为参数$\theta $的似然函数,指样本恰好取观测值的概率

3.设$L(\theta ) = \prod _{i=1}^n p(x_i,\theta )$为参数$\theta $的似然函数,若存在一个只与样本观测值$x_1,x_2,\cdots ,x_n$相关的实数
$\widehat{\theta}(x_1,x_2,\cdots ,x_n) $使得
\[L(\widehat{\theta } )= maxL(\theta )\]
则称$\widehat{\theta}(x_1,x_2,\cdots ,x_n) $为参数$\theta $的最大似然估计值,称$\widehat{\theta}(X_1,X_2,\cdots ,X_n) $为参数$\theta $的最大似然估计量

note:1)似然函数不一定由极大值点,但必然有最大值点,有时候求驻点会失效;2)在多个乘积时,求对数似然函数$\ln L(\theta )$会较好求最大值点,$\ln L(\theta ),L(\theta)$
有相同的最大值点

\subsection{点估计优良性的评判}
不同方法求出的估计量不同时判断哪个估计量更好
\subsubsection{无偏性}
1.若参数$\theta $的估计量$\widehat{\theta } = \widehat{\theta}(X_1,X_2,\cdots ,X_n)$满足
\[E(\widehat{\theta }) = \theta \]
则称$\widehat{\theta }$为$\theta $的无偏估计量

2.解释了为什么方差定义为$S^2 = \frac{1}{n-1}\sum _{i=1}^n(X_i - \overline{X})^2$

3.若求出有偏估计量可以变换为无偏估计量

\subsubsection{有效性}
1.设$\widehat{\theta }_1,\widehat{\theta }_2$,都是参数$\theta $的无偏估计量,如果
\[D(\widehat{\theta }_1) < D(\widehat{\theta }_2)\]
则$\widehat{\theta }_1$比$\widehat{\theta }_2$更有效

\subsubsection{一致性(相合性)}

1.设$\widehat{\theta }_n= \widehat{\theta}(X_1,X_2,\cdots ,X_n)$是$\theta$的一个估计量,对于任意$\varepsilon > 0 $有
\[\lim_{n \to \infty} P\{|\widehat{\theta }_n - \widehat{\theta }| < \varepsilon \} = 1  \] 
则称$\widehat{\theta }_n$是$\theta$的一致估计量

note:这个条件很强,一般很难取到

\end{document}