\documentclass[11pt,twoside,a4paper]{ctexart}
\usepackage[backend=bibtex]{biblatex}
\usepackage[margin=1in]{geometry}
\usepackage{amsmath}

\begin{document}
    \section{磁路}
    \subsection{磁场的基本物理量}
    1.
    \begin{minipage}[t]{0.9\linewidth}
    磁通$\varPhi $ ,单位:Wb

    磁场穿过某个截面的磁感线的条数$\varPhi = BA $
    \end{minipage}


    2.磁感应强度B ,单位:T,矢量,又称磁通密度,掌握右手螺旋定则

    3.
    \begin{minipage}[t]{0.9\linewidth}
    磁场强度H ,单位:A/m, 矢量
    
    方向和B相同,$H\propto i  $
    
    
    表示电流本身产生磁场的强弱
    

    \end{minipage}  

    4.
    \begin{minipage}[t]{0.9\linewidth}
    磁导率$\mu = \tfrac{B}{H} $,单位:H/m,表示介质的导磁能力,值会随着介质的磁化程度变化
    \end{minipage}
    \subsection{物质的磁性能}
    按照磁导率的差异,分为非磁性物质和磁性物质
    \subsubsection{非磁性物质}
    总体特点$\mu  \approx \mu_0 $ 

    其中 $\mu > \mu_0 $ 为顺磁物质,$\mu < \mu_0 $ 为反磁物质

    工业上将所有非磁性物质的磁导率看作$\mu_0 $ ,非磁性物质的B和H成线性关系
    \subsubsection{磁性物质}
    由于磁畴,产生的特殊点

    特点:
    \begin{minipage}[t]{0.9\linewidth}
        1.$\mu \gg \mu_0 $

        2.$\mu $ 会随着磁化程度的变化而变化,从而导致3

        3.B和H成非线性关系

        4.2 3 称为磁饱和性,熟悉初始磁化曲线

        5.磁滞性:B的变化滞后于H,熟悉基本磁化曲线,记忆$H_c $ 矫顽磁力,$B_r $ 剩磁强度

        6.根据磁化曲线的不同,将磁性物质分为三种:硬磁物质,软磁物质,矩磁物质

        \quad 硬磁物质:$H_c $ $B_r $ 较大

        \quad 软磁物质:$H_c $ $B_r $ 较小,用于变压器,电机等的铁心

        \quad 矩磁物质:磁滞回线为矩形,用于电子技术和计算技术中
    \end{minipage}
    \subsection{磁路的基本定律}
    主磁通$\varPhi $:大部分经过铁心闭合的磁通

    漏磁通$\varPhi _\sigma   $:少部分经过空气等非磁性物质的磁通

    磁路:主磁通经过的路径
    \subsubsection{磁路欧姆定律}
    1.恒定磁通

    根据磁通连续性定律,全电流定律

    磁阻$R_m = R_mc + R_mo = \frac{l_c}{\mu _cA_c} + \frac{l_0}{\mu _0A_0} $

    磁通势$F = NI $

    则有
    \begin{equation}
        \varPhi  = \tfrac{F}{R_m}\label{1}
    \end{equation}

    公式 \eqref{1} 是恒定磁场中的欧姆定律,在$R_m = R_mc + R_mo $ 中,$R_mo \gg R_mc $,则在F一定的情况下,磁路中出现空气隙会使$\varPhi $ 减小很多

    2.交变磁通的磁路欧姆定律

    \[ \dot{\varPhi_m } = \frac{\dot{F_m}}{Z_m} \]

    其中$\dot{F_m} $ 为磁通势的幅值,$\dot{F_m} = NI_m = \sqrt{2}NI  $

    $Z_m $为磁阻抗,$Z_m = R_m + jX_m $

    \subsubsection{磁路基尔霍夫定律}
    以恒定磁通磁路为例
    
    1.磁路基尔霍夫第一定律:磁路的一个封闭面流出磁通等于流入磁通,记流出为正流入为负则有$\sum \varPhi = 0 $

    2.磁路基尔霍夫第二定律:在磁路任意一个闭合回路中,磁位差的代数和等于磁通势的代数和$\sum U_m = \sum F $
    \subsection{铁心线圈电路}
    \subsubsection{直流铁心线圈电路}
    直流电流产生恒定磁场,不产生感应电动势,电感相当于短路,所以
    \[I = \frac{U}{R} \]

    线圈消耗的功率只有电阻消耗的功率:
    \[P = I^2R = UI \]
    \subsubsection{交流铁心线圈电路}
    推导见课本

    1.电磁关系:设交流电压u,线圈中通过的电流i,则交流电压在铁心中产生交变磁通,记其中主磁通$\varPhi  $,漏磁通$\varPhi _\sigma  $,主磁通漏磁通在线圈中产生电动势e和$e_\sigma $,线圈电阻R。
    由基尔霍夫电压定律:
    \[u = -e + -e_\sigma + Ri \]
    写做相量
    \[\dot{U} = -\dot{E} - \dot{E_\sigma} + R\dot{I} \]
    由于漏磁通的路径为非磁性物质,则磁导率为常数,$\varPhi _\sigma $和i正比,漏磁通对应的线圈电感
    \[L_\sigma = \frac{N\varPhi _\sigma}{i} \]
    为常数,在交流电路中的电抗称为漏电抗,简称漏抗:
    \[X = \omega L_\sigma = 2\pi fL_\sigma  \]
    又电感的电压和电感电动势相位相反:
    \[\dot{E_\sigma} = -jX\dot{I} \]
    则有电动势平衡方程:
    \[\dot{U} = -\dot{E} + (R + jX)\dot{I} = -\dot{E} + Z\dot{I} \]
    式中:
    \[Z = R + jX \]
    为线圈的漏阻抗。

    主磁通由于磁导率会变化,不能用以上的处理方式,直接用电磁感应定律分析,得到结果
    \[\dot{E} = -j4.44Nf\dot{\varPhi _m} \]
    在忽略R和X时
    \[U = E = 4.44Nf\varPhi _m \]
    说明在电压U和频率f不变时,主磁通$\varPhi _m $几乎不变

    2.功率关系

    交流铁心线圈有功功率:$P = UI\cos \varphi  $,有功功率包括铜损耗和铁损耗

    铜损耗:$P_{Cu} = RI^2 $,就是线圈电阻的功率损耗

    铁损耗:
    \begin{minipage}[t]{0.9\linewidth}
        包括两部分:磁滞损耗$P_h $,涡流损耗$P_e $
       
        1)磁滞损耗:由磁化过程中磁畴相互摩擦造成,经验公式:

        \[P_h = K_hfB_m^\alpha V \]

        $K_h $为磁滞损耗系数,取决于材料,V是铁心体积,$\alpha  $取决于材料。

        选用软磁物质可减小改损耗

        2)涡流损耗:由于涡旋电流产生的损耗,经验公式:

        \[P_e = K_ed^2f^2B_m^2V\]

        $K_e $为磁滞损耗系数,大小和材料的电阻率成反比,d为钢片厚度,V是铁心体积

        选用电阻率大的磁性材料减小该损耗;使用多片硅钢片叠加可以减小该损耗

        铁损耗是两者损耗的和

        \[P_{Fe} = P_h + P_e\]
        
        经验公式:

        \[P_{Fe} = K_{Fe}f^\beta B_m^2m\]

        $K_{Fe} $铁心的损耗系数,$\beta  $频率系数,m铁心质量。恒定磁通的磁路无铁损耗

        工程上采用损耗曲线计算铁损耗

    \end{minipage}
    
    3.等效电路

    目的:将交流铁心线圈电路简化为单纯电路问题

    \begin{align}
        \dot{E} & = -j4.44Nf\dot{\varPhi } =  -j4.44Nf\frac{\dot{F_m}}{\dot{Z_m}} = -j4.44Nf\frac{\sqrt{2}N\dot{I} }{R_m+jX_m}\notag \\
        & = -4.44\sqrt{2}N^2f\Biggl( \frac{X_m}{R_m^2 + X_m^2}+j\frac{R_m}{R_m^2+X_m^2}\Biggr) \dot{I} \notag
    \end{align}
    \[R_0 = 4.44\sqrt{2}N^2f\frac{X_m}{R_m^2 + X_m^2} \]
    \[X_0 = 4.44\sqrt{2}N^2f\frac{R_m}{R_m^2 + X_m^2} \]
    \[Z_0 = R_0 + jX_0 \]

    $R_0 X_0 Z_0 $分别是励磁电阻,励磁电抗,励磁阻抗,则
    \[\dot{E} = -(R_0 + jX_0)\dot{I} = -Z_0\dot{I} \]
    由于U,f不变时$\varPhi _m $基本不变,近似$R_0 X_0 Z_0 $为常数,则电动势平衡方程改写为:
    \[\dot{U} = (R_0 +jX_0)\dot{I} + (R + jX)\dot{I} = (Z + Z_0)\dot{I} \]
    得到交流铁心线圈电路的等效电路图课本图1.4.4

    电流I通过R和$R_0 $的功率是有功功率
    \[P_{Cu} = RI^2 \]
    \[P_{Fe} = R_0I^2\]
    $R_0 $代表铁损耗的等效电阻,X代表漏磁通电感形成的电抗,$X_0 $代表主磁通电感形成的电抗

    产生主磁通的电流为励磁电流,由于$R_0 \gg R , X_0 \gg X $,则可认为I是励磁电流
    


\end{document}




