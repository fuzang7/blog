\documentclass[11pt,twoside,a4paper]{ctexart}
\usepackage[backend=bibtex]{biblatex}
\usepackage[margin=1in]{geometry}
\usepackage{amsmath}

\begin{document}
\section{变压器}
\subsection{变压器的工作原理}
\subsubsection{单相变压器}
原理:线圈称为绕组,工作中,接电源为一次绕组,接负载为二次绕组

1.电压变换 

变压器在工作时,可以变换电压,变化的大小取决于绕组的匝数多少

1)原理:在一次绕组加电压$u_1 $,则在一次绕组中产生电流$i_1 $,$i_1 $在一次绕组中产生主磁通$\varPhi ,\text{漏磁通}\varPhi _{\sigma 1} $,
主磁通$\varPhi $在一二次绕组中分别产生感应电动势$e_1,e_2 $,$e_2 $在二次绕组中产生电流$i_2 $,在负载两端产生电压$u_2 $,$i_2 $参与产生主磁通$\varPhi $和仅与二次绕组铰链的
漏磁通$\varPhi _{\sigma 2} $

一次绕组部分
\[\dot{E_1} = -j4.44N_1f\dot{\varPhi _m} \]
\[\dot{U_1} = -\dot{E_1}+(R_1 + jX_1)\dot{I_1} = -\dot{E_1} + Z_1\dot{I_1} \]
二次绕组部分
\[\dot{E_2} = -j4.44N_2f\dot{\varPhi _m} \]
\[\dot{U_2} = \dot{E_2}-(R_2 + jX_2)\dot{I_2} = \dot{E_2} - Z_2\dot{I_2} \]  
\[\dot{U_2} = Z_L\dot{I_2}  \]
其中$Z_L $是负载部分阻抗

2)电压比:在忽略$Z_1,Z_2 $时为: $k = \frac{E_1}{E_2} = \frac{N_1}{N_2} $.在空载时,即使不忽略$Z_1,Z_2 $,电压比也十分接近匝数比

在求电压比时,用空载时的电压比来计算

2.电流变换

二次电流$I_2 $的大小主要取决于负载阻抗模$|Z_L| $的大小,
一次电流$I_1 $的大小主要取决于$I_2 $的大小。

磁通势平衡方程式:空载时,主磁通由磁通势$\dot{F}_{0m} = N_1\dot{I}_{0m} $产生
有负载时,主磁通由$\dot{F}_{1m} = N_1\dot{I}_{1m},\dot{F}_{2m}  = N_2\dot{I}_{2m} $产生
又由磁路欧姆定律,$U_1 $不变的情况下,主磁通$\varPhi _m $基本不变
\[\dot{F}_{0m} = \dot{F}_{1m} + \dot{F}_{2m} \]
\[N_1\dot{I}_{0m} = N_1\dot{I}_{1m} + N_2\dot{I}_{2m} \]
\[N_1\dot{I}_{0} = N_1\dot{I}_{1} + N_2\dot{I}_{2} \]
上面三个式子等价

在满载时,由于$I_0 $比额定电流小得多,可以忽略,则有
\[\frac{I_1}{I_2} = \frac{N_2}{N_1} = \frac{1}{k} \]
3.阻抗变换

变压器二次绕组阻抗模$|Z_L| $,忽略$ Z_1,Z_2,I_0$,则
\[|Z_L| = \frac{U_2}{I_2} = \frac{U_1/k}{kI_1} = \frac{1}{k^2}\frac{U_1}{I_1} \]
$\frac{U_1}{I_1} $视为从一次绕组看进去的等效阻抗模$|Z_e| $
\[|Z_e| = k^2|Z_L| \]
当负载直接接电源时,电源的负载阻抗$|Z_L| $,通过变压器接电源时,
相当于把阻抗模增加到$|Z_L| $的$ k^2 $倍
\subsubsection{三相变压器}
1.三相变压器的种类:

(1)三相组式变压器:三相之间只有电的联系,没有磁的联系

(2)三相心式变压器:三相之间同时有电与磁的联系

2.三相绕组的联结方式:

\subsection{变压器的基本结构}
\subsubsection{主要部件}
1.铁心:奇数偶数层结构不同,目的减小空气隙

2.绕组:绝缘导线绕成

3.其他:外壳,油箱等等

\subsubsection{主要种类}
1.按照结构分类:心式变压器和壳式变压器

心式变压器:铜包铁,壳式变压器:铁包铜
\subsubsection{额定值}
1.额定电压$U_{1N} /U_{2N} $,指空载电压的额定值,
三相变压器的额定电压指在空载时高低压绕组线电压的额定值

2.额定电流$I_{1N} /I_{2N} $,指满载电压的额定值,
三相变压器的额定电流时满载时高低压绕组线电流值

工作电流若长期超过额定电流,变压器温度会过高

3.额定容量$S_N $,指变压器视在功率的额定值

单相变压器:$S_N = U_{2N}I_{2N} = U_{1N}I_{1N} $

三相变压器:$S_N =\sqrt{3}U_{2N}I_{2N} = \sqrt{3}U_{1N}I_{1N} $

4.额定频率$f_N $:50Hz

变压器的电压电流功率频率都在额定值时,称为额定状态,此时的电流为额定值,
就是满载状态
\subsection{变压器的运行分析}
\subsubsection{等效电路}
1.折算:将匝数为$N_2 $的二次绕组折算为匝数为$N_1$,的绕组,在这个过程中要求磁通势不变
功率不变,将折算后的物理量右上角加" ' "

2.以低压绕组向高压绕组这算为例,计算物理量的变化

1)二次绕组电流$I_2' $
\[I_2' = \frac{I_2}{k} \]

2)二次绕组的电压$U_2' $电动势$E_2' $
\[U_2' = kU_2 \]
\[E_2' = kE_2\]

3)二次绕组漏阻抗$Z_2' $负载阻抗$Z_L' $
\[Z_2' = k^2Z_2 \]
\[Z_L' = k^2Z_L \]

3.T型等效电路:p32

4.简化等效电路:p32,在电流满载或接近时使用




\subsubsection{基本方程式}
折算前
\[\begin{cases}
    
\end{cases}\]
折算后
\subsubsection{相量图}
由T型等效电路和基本方程式得
\subsection{变压器的参数测定}
\subsubsection{空载试验}
试验在低压绕组进行(目的是可测量),将低压绕组作为一次绕组,高压绕组作为二次绕组且输出端开路

1.铁损耗$P_{Fe} $
在空载试验中,由于$I_0 \ll I_1$,一次绕组部分铜损$P_{Cu} = RI^2 $很小,二次绕组部分$I_2 = 0 $,没有铜损
所以可以忽略铜损。而铁损耗主要取决于一次绕组的电压和频率,在空载时都没有发生变化,则认为空载试验功率$P_0 $有:
\[P_0 = P_{Fe} \]

2.励磁阻抗模$|Z_0| $

由T型等效电路的空载时:
\[\frac{U_1}{I_0} = |Z_1 + Z_0| \]
又
\[Z_1 \ll Z_0 \]
则:
\[|Z_0| = \frac{U_1}{I_0} \]

3.励磁电阻$R_0 $
\[R_0 = \frac{P_0}{I_0^2} \]

4.励磁电抗$ X_0 $
\[X_0 = \sqrt{|Z_0|^2 - R_0^2}\]

5.电压比k
\[k = \frac{U_2}{U_1}\]
由于空载试验在低压侧进行,所以上述参数$|Z_0|, R_0,X_0$是折算到低压侧数值,
当高压绕组作为一次绕组时,需要用高压侧的数值,将上述参数乘以$K^2 $则可
\subsubsection{短路试验}
试验在高压侧进行,将高压绕组作为一次绕组,将低压绕组的输出端短路,
电压逐渐从零增加到电流等于额定电流为止,p36图

1.铜损耗
由于输出端短路,所以$U_s$很小,所以铁损也很小,而铜损耗是满载铜损耗,所以认为可以忽略铁损耗
\[P_{Cu} = P_S \]
2.短路阻抗模$|Z_S| $
\[|Z_S| = \frac{U_S}{I_1}\]

3.短路电阻$R_S $
\[R_S = \frac{P_S}{I_1^2}\]

4.短路电抗$X_S $
\[X_S = \sqrt{|Z_S|^2 - R_S^2}\]

5.阻抗电压$U_S $

指额定电流通过短路阻抗时的电压,

阻抗电压的标幺值:
\[\text{标幺值} = \frac{\text{实际值}}{\text{基值}} \]
基值取对应的额定值,标幺值用右上角的*表示
\[U_s^* = \frac{U_S}{U_{1N}} = \frac{|Z_S|I_1N}{U_{1N}} = Z_S^*\]
6.温度转换

要将绕组电阻转换为75度的电阻

由于空载试验在高压侧进行,所以上述参数$|Z_0|, R_0,X_0$是折算到高压侧数值,
当低压绕组作为一次绕组时,需要用低压侧的数值,将上述参数除以$K^2 $则可
\subsection{变压器的运行特性}
\subsubsection{外特性}
1.定义:保持电压$U_1,\lambda _2$不变,二次电压和电流之间的关系$U_2 = f(I_2)$为变压器的外特性

2.特点:电感性电阻性电容性负载的外特性有所不同,其中电容性负载较为特殊,随着$I_2$增大,$U_2$增大,
而电感性负载为$\lambda _2 $越低,$U_2$下降越多

3.电压调整率:用来表示$U_2$随$I_2$变化的程度。在一次电压为额定值,功率因数不变的情况下,变压器从满载到空载
二次电压变化的数值和空载电压的比值
\[V_R = \frac{U_{2N} - U_2}{U_{2N}}*100\%\]
折算到一次侧:
\[V_R = \frac{U_{1N} - U_2'}{U_{1N}}*100\%\]
利用简化电路(在满载和接近满载时使用)
\[V_R = (R_S\cos\varphi _2 + X_S\sin\varphi _2)\frac{I_{1N}}{U_{1N}}*100\%\]

负载电感性:$\varphi _2 > 0$;电阻性:$\varphi _2 = 0$;电容性:$\varphi _2 < 0$
\subsubsection{效率特性}
1.功率:变压器由电源输出的有功功率
\[P_1 = U_1I_1\cos\varphi _1\]
变压器输出的有功功率
\[P_2 = U_2I_2\cos\varphi _2\]

2.损耗:总损耗
\[P_1 - P_2 = P_{Cu} + P_{Fe}\]
\begin{align}
    P_{Cu} & = R_1I_1^2 + R_2I_2^2 \notag \\
    & = R_1I_1^2 + R'_2I'_2{}^2 \text{ T形等效} \notag \\
    & = R_SI_1^2 = R_SI'_2{}^2 \text{ 简化等效} \notag \\
    & = \beta ^2P_S \text{ 短路试验}
\end{align}

其中
\[\beta = \frac{I_1}{I_{1N}}\]
所以铜损会变,称为可变损耗
\begin{align}
    P_{Fe} & = R_0I_0^2 \text{ T形等效} \\
    & = P_0 \text{ 空载试验}
\end{align}
由于铁损和$I_2$无关,称为不变损耗
3.效率
\[\eta = \frac{P_2}{P_1} * 100\%\]
在忽略$V_R$
\[\eta = \frac{\beta S_N\lambda _2}{\beta S_N\lambda _2 + P_0 + \beta ^2P_S} * 100\%\]

4.效率特性:
当$U_1 = U_{1N},\cos \varphi _2$常数时,$\eta = f(I_2) $或$\eta = f(\beta)$
有求导求$\eta$的最大值,在$P_{Cu} = P_{Fe}$时,有最大值
\subsection{三相变压器的联结组}

\end{document}
\subsection{三相变压器的联结组}