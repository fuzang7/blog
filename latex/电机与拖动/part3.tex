\documentclass[11pt,twoside,a4paper]{ctexart}
\usepackage[backend=bibtex]{biblatex}
\usepackage[margin=1in]{geometry}
\usepackage{amsmath}

\begin{document}
    \section{直流电机的基本理论}
    \subsection{直流电机的原理}
    \subsubsection{直流电动机原理}

    1.电枢:绕轴转动的绕组

    2.换向片:铜片,换向过程中使用

    3.换向器:换向片组合

    4.做功关系:直流通过换向变成交流,克服反电动势,旋转
    \subsubsection{直流发电机原理}
    基础结构相同,但是功能关系不同
    1.做功关系:电机向负载输出电功率,动机向电机输出机械功率。
    \subsection{直流电机的原理}
    \subsubsection{主要部件}
    1.定子:
    \begin{minipage}[t]{0.9\linewidth}
        1)主磁极:励磁绕组绕在主磁极铁心上,下方有极靴用来使气隙中磁场分布均匀,起到固定励磁绕组的作用

        2)换向磁极:和电枢绕组串联,在换向过程中抵消电流变化产生的电动势从而保护线路

        3)机座

        4)电刷装置

        5)端盖
    \end{minipage}

    2.转子:
    \begin{minipage}[t]{0.9\linewidth}
        1)电枢铁心:

        2)电枢绕组:相邻两个线圈的端部接到同一个换向片上,
        不同的线圈之间为串并联关系

        3)换向器:

        4)转轴:

        5)风扇:
    \end{minipage}

    \subsubsection{励磁方式}
    他励,并励,串励,复励,后三个同是自励

    以电枢(电动机与发电机)和励磁绕组的供电方式分类:

    他励:两部分分开供电

    并励:两部分同一个电源供电,且并联

    串励:两部分同一个电源供电,且串联

    复励:有两个励磁绕组,一个串一个并(两组串并组合都可以)

    \subsection{额定值}
    1.额定电压$U_N$
    
    发电机$U_N$:输出电压额定值

    电动机$U_N$:输入电压额定值

    2.额定电流$I_N$

    发电机$I_N$:输出电流额定值

    电动机$I_N$:输入电流额定值

    3.额定功率$P_N$

    发电机$P_N$:输出电功率的额定值$(U_NI_N)$

    电动机$P_N$:输出机械功率的额定值$(U_NI_N\eta _N)$

    4.额定转速$n_N$

    5.额定励磁电压$U_{fN}$

    6.额定励磁电流$I_{fN}$

    额定状态:UIPn全额定

    满载状态:$I = I_N$

    \subsection{直流电机的电枢反应}
    \subsubsection{原理}
    直流电机工作时的磁通势

    1.励磁磁通势:由励磁绕组通电流产生的的磁通势,在气隙之间磁感应强度基本相等

    2.电枢磁通势:有电枢绕组通电枢电流产生

    上述两磁通势合成为电机的工作总磁场,在电机负载变化是,电枢电流变化,电枢磁通势变化,使得合成磁通势也变化,
    电枢磁通势对合成磁通势的影响称为电枢反应
    \subsubsection{影响}
    1.磁场呗扭歪,使得磁场的物理中心线和几何中心线分开,在电刷短路时换向线圈
    电动势不为零,增加了换向困难

    2.一半磁极磁通增加,一半磁极磁通减少。磁路不饱和时,增加与减少的磁通相同,每极磁通不变。
    磁路饱和时,增加的少,减少的多,每极磁通减少,电动势和电磁转矩随之减小

\end{document}