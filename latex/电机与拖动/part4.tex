\documentclass[11pt,twoside,a4paper]{ctexart}
\usepackage[backend=bibtex]{biblatex}
\usepackage[margin=1in]{geometry}
\usepackage{amsmath}
\begin{document}
\section{直流电机的电力拖动}
\subsection{他励直流电动机的机械特性}
机械特性:在$U_a,R_a,I_f$不变时,电动机转速n和电磁转矩T之间的关系
\begin{align}
    n &= \frac{U_a}{C_E\varPhi } - \frac{R_a}{C_EC_T\varPhi ^2}T  \label{root}      \\
    & = n_0 - \gamma T \notag \\
    & = n_0 - \Delta n   \notag
\end{align}

上式中,$n_0$为理想空载转速,$\gamma = |\frac{dn}{dT}| = \frac{R_a}{C_EC_T\varPhi ^2}$为机械特性的斜率
机械特性的硬度$\alpha  = \frac{1}{\gamma }$

\subsubsection{固有特性}
固有特性:$U_a,I_f$保持额定值,电枢电路中没有外接电阻时的机械特性

\[n = f(T)|_{U_{aN},I_{fN},R_a}\]

在p213图像上,左侧的N点位电动机的额定状态,能长期运行;右侧的M点为临界状态,可以短时间过载,过载能力$\alpha _{MC} = \frac{I_{amax}}{I_{aN}}$

\subsubsection{人为特性}
根据 \eqref{root}公式分析

1.增加电枢电阻时的人为特性
$R_a$增加,引起变化$n_0$不变,$\gamma $增加,$\alpha $减小

2.降低电枢电压时的人为特性
$U_a$降低,引起变化$n_0$降低,$\gamma,\alpha $不变

3.减小励磁电流时的人为特性
$I_a$降低,引起变化$\varPhi $降低,$n_0,\gamma$增加,$\alpha $减小

\subsection{他励直流电动机的起动}
起动电流:直流电动机电源刚接通瞬间,转子没有转动时,电动机的输入电流

起动转矩:起动瞬间的电磁转矩

起动电流的分析:他励电动机起动瞬间,n = 0,E = 0,则起动电流$I_S = \frac{U_a}{R_a}$
在额定电压下起动时,由于电阻$R_a$太小,所以起动电流很大,能达到额定电流的十倍以上,
造成起动转矩也达到额定转矩的十倍以上,极容易造成电动机的损坏。除了微型电动机,不然不会允许
直流电动机直接起动。

为了安全起动,需要限制起动电流,也就是减小$U_a$或增大$R_a$

\subsubsection{降低电枢电压起动}
额外使用一台可以调节电压的专用直流电源,在起动时,施加额定的励磁电压励磁电流$U_f,I_f$,让电枢电压从0逐渐增加到
额定值

优点:能耗小,平稳

缺点:贵

\subsubsection{增加电枢电阻起动}
1.无极起动

对于额定功率较小的电动机,在电枢电路内串联起动变阻器,在起动时,施加额定的励磁电压励磁电流$U_f,I_f$,
让起动变阻器为最大电阻,随着转速的增加,让起动变阻器的大小逐渐减小,直到全部切除

起动变阻器最大值$R_{ST} = \frac{U_a}{I_S} - R_a;\text{忽略}T_0,R_a = \frac{U_{aN} - \frac{P_N}{I_{aN}}}{I_{aN}}$
由于忽略,估算$R_a$偏大

2.有级起动



\subsection{他励直流电动机的调速}
由
\[n &= \frac{U_a}{C_E\varPhi } - \frac{R_a}{C_EC_T\varPhi ^2}T\]
改变$R_a,U_a,\varPhi $都能改变转速

\subsubsection{改变电枢电阻调速}

\subsubsection{改变电枢电压调速}

\subsubsection{改变励磁电流调速}

\subsection{他励直流电动机的制动}

\subsubsection{能耗制动}

1.迅速停机时的能耗制动

2.下放重物时的能耗制动

\subsubsection{反制制动}

1.迅速停机时的反制制动

2.下放重物时的反制制动

\end{document}