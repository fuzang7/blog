\documentclass[11pt,twoside,a4paper]{ctexart}
\usepackage[backend=bibtex]{biblatex}
\usepackage[margin=1in]{geometry}
\usepackage{amsmath}
\begin{document}
    
\section{异步电机基本理论}
\subsection{异步电机的用途和种类}

由于性能,主要作为电动机使用,且三相的相对多

\subsection{三相异步电动机工作原理}
利用三相电流通过三相绕组产生空间旋转磁场

\subsubsection{旋转磁场}
1.旋转磁场的产生

由三相电流通过三相绕组或者多相电流通过多相绕组产生,产生的合成磁场在空间是旋转的
不同结构的绕组磁场会有不同的旋转速度

机械角度:磁场的空间转过的角度

电磁角度:电流变化的角度

则

电磁角度 = 磁极对数 * 机械角度

2.旋转磁场的转速

旋转磁场的转速为同步转速$n_0$,单位$r/min$
\[n_0 = \frac{60f_1}{p}\]

3.旋转磁场的转向

与三相绕组中三相电流的相序一致

\subsubsection{工作原理}
1.电磁转矩的产生

转子绕组和旋转磁场存在相对运动,所以绕组切割磁感线产生交流感应电动势,
然后产生感应电流,
感应电流可以分为有功分量和无功分量。
其中有功分量和感应电动势相位相同,方向相同
产生电磁力F,形成电磁转矩

转子绕组的感应电流无功分量在相位上滞后于感应电动势$\pi /2$
产生的电磁力相互抵消,不会产生电磁转矩

note:转子转速总是小于同步转速,故而叫做异步电动机


2.转差率

\[s = \frac{n_0 - n}{n_0}\]
用来反映转子和旋转磁场相对运动速度的大小

3.电机的不同工作状态
\begin{minipage}[t]{0.9\linewidth}
    1)堵转:电机刚接通电源,还未转动时。$n = 0,s = 1$

    2)理想空载:转子转速与同步转速相等时。$n = n_0,s = 0$

    3)电动机状态:电机作为电动机工作。$0<n<n_0,0<s<1$

    4)发电机状态:转子转速超过同步转速。$n>n_0,s<0$

    5)制动:转子转向和旋转磁场转向相反。$n<0,s>1$
    
\end{minipage}

4.电磁转矩的大小

\[T = C_T\varPhi I_2 \cos \varphi _2\]
其中$C_T$有电机结构决定,视为常数

5.电磁转矩的方向

电磁转矩方向和旋转磁场转向相同,也与三相绕组中三相电流相序一致

\subsection{三相异步电动机的基本机构}

\subsubsection{主要部件}
1.定子:1)定子铁心;2)定子绕组;3)机座;4)端盖

2.转子:1)转子铁心;2)转子绕组:根据结构分为笼型,绕线型;3)转轴;4)风扇
\subsubsection{三相绕组}
定子绕组和绕线型转子绕组都是三相绕组,绕组由若干线圈连接而成,两个线圈边分别嵌放在两个贴心槽内

单层绕组:一个槽只嵌放一个圈边

双层绕组:一个槽嵌放两个分属两个线圈的圈边,一个内层一个外层

1.三相单层绕组

p69图 三相单层绕组的接线图,记槽数为z

1)槽间距:相邻两槽中心线间的电角度
\[\alpha = \frac{p*360^\circ}{z}\]
其中p为磁极对数,z为电机槽数

2)极距$\tau $:相邻两磁极中心线的距离
\[\tau = \frac{z}{2p}\]

3)线圈的节距y:线圈两圈边之间的距离

4)线圈的相数m

5)每极每相槽数q:每相绕组在每个极内所占槽数
\[q = \frac{z}{2pm}\]

6)三相绕组的组成
p70

2.三相双层绕组

我看不懂,但我大受震撼

3.三相绕组的种类
按照线圈节距y不同分为:1)$y = \tau $整距绕组;2)$y < \tau $短距绕组

按照每极每相槽数q不同分为:1)$q = 1$集中绕组;2)$q > 1$分布绕组

分布绕组每极每相的q个线圈边是嵌放在沿圆周分布的相邻槽内,集中绕组则相当于每极每相的所有线圈边都集中在一个槽内

\subsubsection{笼型绕组}

1.极数:
\[p_1  = p_2 = p\]

2.相数$m_2$:1)当转子齿数$z_2$能被磁极对数p整除时
\[m_2 = \frac{z_2}{p}\]

2)当当转子齿数$z_2$不能被磁极对数p整除时
\[m_2 = z_2\]

3.匝数
$N_2 = \frac{1}{2}$
\subsection{三相异步电动机的额定值}

1.额定功率$P_N$:简称容量,指电动机在额定状态下时的输出机械功率

2.额定电压$U_N$:额定状态时的三线绕组的线电压

3.额定电流$I_N$:额定状态时的三线绕组的线电流

\subsection{三相异步电动机的电动势平衡方程式}

\subsubsection{定子电路的电动势平衡方程式}

1.三相异步电动机定子每相电路的电动势平衡方程式:
\[\dot{U}_1 = -\dot{E}_1 + (R_1 + jX_1)\dot{I}_1 = -\dot{E}_1 + Z_1\dot{I}_1\]
其中$R_1,X_1,Z_1,U_1,I_1,E_1 $是定子每相绕组的电阻,漏电抗,漏阻抗,相电压,相电流,相电动势

2.变压器一二次绕组是集中绕在铁心上,相当于整距集中绕组。

3.绕组因数$k_w$:在匝数,磁通最大值相同的情况下,短距分布绕组的电动势小于整距集中绕组中的,两者之比为绕组因数

4.有效匝数:$k_wN$

5.定子每相绕组中的感应电动势$E_1 = 4.44k_{w1}N_1f_1\varPhi _m$,其中$varPhi _m$是旋转磁场磁通的幅值

6.定子频率$f_1 = \frac{pn_0}{60}$,定子感应电动势的频率和定子电压和电流的频率相同

7.若忽略$R_1,X_1$
\[U_1 = E_1 = 4.44k_{w1}N_1f_1\varPhi _m\]
得$\varPhi _m$

\subsubsection{转子电路得电动势平衡方程式}

1.转子电路是短路的,所以$U_2 = 0$,则电动势平衡方程式
\[0 = \dot{E}_{2s} - (R_2 + jX_{2s})\dot{I}_{2s} = \dot{E}_{2s} - Z_{2s}\dot{I}_{2s}\]
\[X_{2s} = 2\pi f_2L_2\]
\[E_{2s} = 4.44k_{w2}N_2f_2\varPhi _m\]
在其中$k_{w2}$是转子绕组的绕组因数,如果转子绕组是笼型绕组,则每相只有一根导线,$k_{w2} = 0$

2.$X_{2s},E_{2s}$都是和转差率s相关的,因为$f_2 = sf_1$,则
\[X_{2s} = sX_2\]
\[E_{2s} = sE_2\]

\subsubsection{绕组因数}
绕组因数是节距因数和分布因数的乘积$k_w = k_p*k_s$

1.节距因数$k_p$:短距绕组的电动势和整距绕组的电动势之比称为绕组的节距因数
\[k_p =\sin \frac{y}{\tau }90^\circ \]
其中$y,\tau $为节距和极距

在整距绕组中$k_p = 1$,在短距绕组中$k_p < 1$

2.分布因数$k_s$:分布绕组的电动势和集中绕组的电动势之比称为绕组的分布因数
\[k_s = \frac{\sin \frac{q\alpha }{2}}{q\sin \frac{\alpha}{2}}\]
其中$q,\alpha$分别是一个槽中线圈的个数,相量图中电动势之间的角度

在集中绕组中$k_s = 1$,在分布绕组中$k_s < 1$

\subsection{磁通势平衡方程式}

\subsubsection{磁通势平衡方程式}

1.定子三相电流通过三相绕组时产生的定子旋转磁通势的幅值
\[F_{1m} = \frac{0.9m_1k_{w1}N_1I_1}{2p}\]
旋转方向和定子电流相序一定,转速$n_1 = n_0$

2.转子多相电流通过转子多相绕组时产生的转子旋转磁通势的幅值
\[F_{2m} = \frac{0.9m_2k_{w2}N_2I_{2s}}{2p}\]
旋转方向和转子电流相序一致

相对于转子自身的转速
\[n_2 = \frac{60f_2}{p} = \frac{60sf_1}{p} = sn_0\]

在空间中的转速
\[n_2 + n = sn_0 + (1 - s)n_0 = n_0\]

3.转子旋转磁通势和定子旋转磁通势在空间里以相同的转速同一方向旋转

4.理想空载时,$I_{2s} = 0,I_1 = I_0$,这时旋转磁通的幅值
\[F_{0m} = \frac{0.9m_1k_{w1}N_1I_0}{2p}\]

5.接入负载时,磁通势平衡方程式依然满足
\[\dot{F}_{1m} + \dot{F}_{2m} = \dot{F}_{0m}\]

\[m_1k_{w1}N_1\dot{I_1} + 9m_2k_{w2}N_2\dot{I_{2s}} = [m_1k_{w1}N_1\dot{I_0}\]

\subsubsection{脉振磁通势}
1.脉振磁场:单相电流通过单相绕组产生的方位不变而大小和方向随时间按正弦规律变换的磁场

2.脉振磁通势:脉振磁场产生的磁通势

3.对于磁极对数为p的磁场,气隙磁通势等于绕组磁通势的$\frac{1}{2p}$

4.用傅里叶级数讲气隙磁通势分解,会得到基波和各种高次谐波,气隙基波磁通的幅值
\[F_{pm} = \frac{4}{\pi }*\frac{1}{2p}k_wNI_m = \frac{2\sqrt 2}{\pi p}k_wNI = \frac{0.9k_wNI}{p}\]

\subsubsection{旋转磁通势}
1.三相电流通过三相绕组时,三个绕组的三个基波脉振磁通势
\[F_{pU} = F_{pm}\sin \omega t\] 
\[F_{pV} = F_{pm}\sin (\omega t - 120^\circ )\]
\[F_{pW} = F_{pm}\sin (\omega t + 120^\circ )\]

2.分析得到结论:合成磁通势在空间中时旋转的,且任意时间合成磁通势的幅值为$\frac{3}{2}F_{pm}$

3.若相数为m,则m相绕组通过m相电流产生的基波旋转磁通势的幅值为
\[F_m = \frac{0.9mk_wNI}{2p}\]

\subsubsection{高次谐波磁通势}
1.由于对称性,脉振磁通势中只含有奇次谐波,其中$v = 1$的是基波,$x = 3,5,\cdots $的是高次谐波
高次谐波脉振磁通势将合成高次谐波旋转磁通势,产生高次谐波旋转磁场,从而在转子和定子中产生高次谐波
感应电动势

2.高次谐波感应电动势
\[E_v = 4.44k_{wv}Nf_v\varPhi _{mv}\]

3.磁极对数:$p_v = vp_1$

4.频率:$f_v = vf_1$

5.节距因数:$k_{pv} = \sin v\frac{y}{\tau }90^\circ $

6.分布因数:$k_{sv} = \frac{\sin vq\frac{\alpha}{2}}{q\sin v\frac{\alpha}{2}} $

7.谐波磁通势是有害的,用短距分布绕组可以有效削弱

8.高次谐波磁场会产生寄生转矩,笼型异步电动机转子左侧斜槽来减弱此影响



\end{document}