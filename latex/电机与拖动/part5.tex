\documentclass[11pt,twoside,a4paper]{ctexart}
\usepackage[backend=bibtex]{biblatex}
\usepackage[margin=1in]{geometry}
\usepackage{amsmath}
\begin{document}
    
\section{异步电机基本理论}
\subsection{异步电机的用途和种类}

由于性能,主要作为电动机使用,且三相的相对多

\subsection{三相异步电动机工作原理}
利用三相电流通过三相绕组产生空间旋转磁场

\subsubsection{旋转磁场}
1.旋转磁场的产生

由三相电流通过三相绕组或者多相电流通过多相绕组产生,产生的合成磁场在空间是旋转的
不同结构的绕组磁场会有不同的旋转速度

机械角度:磁场的空间转过的角度

电磁角度:电流变化的角度

则

电磁角度 = 磁极对数 * 机械角度

2.旋转磁场的转速

旋转磁场的转速为同步转速$n_0$,单位$r/min$
\[n_0 = \frac{60f_1}{p}\]

3.旋转磁场的转向

与三相绕组中三相电流的相序一致

\subsubsection{工作原理}
1.电磁转矩的产生

转子绕组和旋转磁场存在相对运动,所以绕组切割磁感线产生交流感应电动势,
然后产生感应电流,
感应电流可以分为有功分量和无功分量。
其中有功分量和感应电动势相位相同,方向相同
产生电磁力F,形成电磁转矩

转子绕组的感应电流无功分量在相位上滞后于感应电动势$\pi /2$
产生的电磁力相互抵消,不会产生电磁转矩

note:转子转速总是小于同步转速,故而叫做异步电动机


2.转差率

\[s = \frac{n_0 - n}{n_0}\]
用来反映转子和旋转磁场相对运动速度的大小

3.电机的不同工作状态
\begin{minipage}[t]{0.9\linewidth}
    1)堵转:电机刚接通电源,还未转动时。$n = 0,s = 1$

    2)理想空载:转子转速与同步转速相等时。$n = n_0,s = 0$

    3)电动机状态:电机作为电动机工作。$0<n<n_0,0<s<1$

    4)发电机状态:转子转速超过同步转速。$n>n_0,s<0$

    5)制动:转子转向和旋转磁场转向相反。$n<0,s>1$
    
\end{minipage}

4.电磁转矩的大小

\[T = C_T\varPhi I_2 \cos \varphi _2\]
其中$C_T$有电机结构决定,视为常数

5.电磁转矩的方向

电磁转矩方向和旋转磁场转向相同,也与三相绕组中三相电流相序一致

\subsection{三相异步电动机的基本机构}

\subsubsection{主要部件}
1.定子:1)定子铁心;2)定子绕组;3)机座;4)端盖

2.转子:1)转子铁心;2)转子绕组:分为笼型,绕线型;3)转轴;4)风扇

\end{document}