\documentclass[11pt,twoside,a4paper]{ctexart}
\usepackage[backend=bibtex]{biblatex}
\usepackage[margin=1in]{geometry}
\usepackage{amsmath}

\begin{document}
    
\section{异步电机的电力拖动}

\subsection{三相异步电动机的机械特性}

\subsubsection{电磁转矩公式}

1.电磁转矩物理公式
\[T = C_T\varPhi _m I_2 \cos \varphi _2\]
其中$C_T$为转矩常数,由电机结构决定

2.电磁转矩的参数公式
\[T = K_T\frac{spR_2U_1^2}{f_1[R_2^2 + (sX_2)^2]}\]

3.电磁转矩的实用公式

1)临界转差率
\[s_M = \frac{R_2}{X_2}\]

2)最大电磁转矩
\[T_M = K_T\frac{pU_1^2}{2f_1X_2}\]

3)忽略空载转矩,额定电磁转矩等于额定输出转矩
\[T_N = T_{2N} = \frac{60}{2\pi}\frac{P_N}{n_N}\]

4)实用公式
\[\frac{T}{T_M} = \frac{2}{\frac{s}{s_M} + \frac{s_M}{s}}\]

\subsubsection{固有特性}

1.转矩特性:$U_1,f_1,R_2,X_2$不变,T与s的关系$T = f(s)$

2.机械特性:n与T的关系$n = f(T)$

3.固有特性:定子电压和频率都是额定值,且为绕线型异步电动机,转子不另外串联电阻或电抗,这时的转矩特性和机械特性为固有特性,否则为人为特性

4.额定状态:工作点在特性曲线上为N点,说明了电动机的长期运行能力。

\ 1)硬特性:转矩增加转速下降不多的机械特性

\ 2)软特性:转矩增加转速下降很多的机械特性

5.临界状态:电动机的电磁转矩最大时的状态,说明了电动机的短时过载能力

1)堵转:负载转矩大于最大转矩导致转速逐渐下降为0,会导致电流远大于额定电流,可能导致电动机严重过热

6.堵转状态:电动机刚接通电流还没有转动的状态,工作点在特性曲线上为S点,转差率$s = 1$,转速$n = 0$,对应的电磁转矩$T_S$为堵转转矩,定子线电路为堵转电流,
此状态说明了电动机的直接起动能力

\ 1)起动转矩倍数$\alpha _{ST}$:
\[\alpha _{ST} = \frac{T_S}{T_N}\]

\ 1)起动电流倍数$\alpha _{SC}$:
\[\alpha _{SC} = \frac{I_S}{I_N}\]

\subsubsection{人为特性}

1.降低定子电压时的人为特性

2.降低转子电阻时的人为特性

\subsection{电力拖动系统的稳定运行}

\subsubsection{负载的机械特性}

负载的机械特性:电动机负载的转速与负载转矩的关系$n = f(T_L)$,简称负载特性

1.恒转矩负载特性:负载转矩为定值,与转速无关

\ 1)反抗性恒转矩负载:负载转矩由摩擦作用产生,其绝对值不变,作用的方向总是与旋转方向相反

\ 2)位能性恒转矩负载:负载转矩由重力作用产生,负载转矩的大小和方向都不变

2.恒功率负载特性:负载转矩的大小与转速的大小成反比,两者的乘积为常数

3.通风机负载特性:负载转矩的大小和转速的平方成正比$T_L\propto  n^2$,负载转矩的方向始终与转速方向相反

\subsubsection{稳定运行的条件}

稳定运行时,必须为$T_2 = T_L$,若是忽略$T_0$,则要有$T = T_L$,同时要求运行要具有一定的抗干扰能力,平衡被打破后能恢复

1.电力拖动系统稳定运行的条件是:电动机的机械特性与生成机械的负载特性由交点,而且在该焦点处满足
\[\frac{dT}{dn} < \frac{dT_L}{dn}\]

\subsection{三相异步电动机的起动}

\subsubsection{电动机的起动指标}

起动是电动机在接通电源后,转子从静止状态开始转动直到稳定运行的过程,由两个基本要求

1.起动转矩要大:$T_S > T_L$时电动机才能起动

2.起动电流(线电流)不能超过允许范围

\subsubsection{笼型异步电动机的直接起动}

在定子绕组上直接加上额定电压起动

\subsubsection{笼型异步电动机的减压起动}

在起动时先降低定子绕组上的电压,在起动后恢复

1.定子串联电阻或电抗减压起动:能耗较大,

2.星形-三角形减压起动:仅适用于正常工作时是三角形联结的电动机,在起动时定子绕组按星形联结,起动后换为三角形连接。
电动机的起动电流,电源电流和起动转矩只有直接起动的三分之一

3.自耦变压器减压起动:同时适用于正常工作时为星形或三角形联结的电动机,电动机本身的起动电流减小至直接起动的$K_A$倍,电源电流和起动转矩都为直接起动的$K_A^2$倍

4.软起动器起动

\subsubsection{绕线型异步电动机转子电路串联电阻起动}

可以同时减小起动电流增加起动转矩

1.无极起动:转子电路串联起动变阻器,起动变阻器最大值
\[R_{ST} = (\frac{T_N}{s_NT_1} - 1)R_2\]
其中$T_1,R_2$为所要求的起动转矩值,转子每相绕组的电阻

2.有机起动:起动电阻为串联的多个电阻,起动瞬间接入最大电阻,随着转速增加,转矩下降为切换转矩$T_2$时,切除一段电阻,转矩恢复$T_1$,重复直到所有电阻都被切除

1)起动电阻的计算:
\begin{minipage}[t]{0.9\linewidth}

    1.选择起动转矩和切换转矩$T_1 = (0.8 \sim 0.9)T_M,T_2 = (1.1 \sim 1.2)T_L$

    2.求起切转矩比$\beta = \frac{T_1}{T_2}$

    3.确定起动级数
    \[m = \frac{\log \frac{T_N}{s_NT_1}}{\log \beta }\]
    m取相近的整数

    4.重新计算$\beta $,校对$T_2$是否在规定范围内

    5.求出各级起动电阻
    \[R_{STi} = (\beta ^i - \beta ^{i - 1})R_2\]

\end{minipage}

\subsection{三相异步电机的调速}

\subsubsection{电动机的调速指标}

1.调速范围:电动机在满载情况下所能得到的最高转速和最低转速的比
\[D = \frac{n_{max}}{n_{min}} = n_{max} : n_{min} \]

2.调速方向:上调和下调

3.调速的平滑性

\ 1)有极调速:在调速中两个转速之间的转速无法取到

\ 2)无极调速:一定范围内的转速都能取到

\ 3)平滑系数:两相邻转速比$\sigma = \frac{n_i}{n_{i - 1}}$

4.调速的稳定性:电动机在新转速下运行,负载变化引起转速变化的程度

\ 1)静差率:某一机械特性上运行时,电动机由理想空载到满载时转速差与里想空载转速的百分比
 \[\delta = \frac{n_0 - n_1}{n_0}*100\% \]
 $\delta$ 越小稳定性越好
 
 \ 2)机械硬度:
  \[\alpha = |\frac{dT}{dn}|\]
  机械硬度越大静差率越小

5.调速时的允许负载

1)恒功率调速:电动机在各种不同的转速下满载运行,允许输出的功率相同

2)恒转矩调速:电动机在各种不同的转速下满载运行,允许输出的转矩相同

三相异步电动机
\[n = (1 - s)n_0 = (1 - s)\frac{60f_1}{p}\]
则调速方法分为变极调速与变频调速

\subsubsection{笼型异步电动机的变频调速}

1.$f_1 < f_N$时,要保持$\frac{U_1}{f_1} \approx \frac{E_1}{f_1} = $常数

$U_1 \approx E_1 = 4.44k_{w1}N_1f_1\varPhi _m$,单独降低频率会使$\varPhi _m$增加,导致磁路饱和,铁损增加,功率因数下降,则$U_1$应该随频率一起下降

2.$f_1 > f_N$时,要保持$U_1 = U_N = $常数

3.变频调速的性能:
\begin{minipage}[t]{0.9\linewidth}
    
    1)调速可以上下调

    2)平滑,可以实现无极调速

    3)稳定性好

    4)调速范围大

    5)$f_1 < f_N$时为恒转矩调速,$f_1 > f_N$时为恒功率调速

\end{minipage}


\end{document}